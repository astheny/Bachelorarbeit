%PAKETE

\usepackage{amsmath,amssymb,amsfonts}
% Wieder auskommentieren wenn bbm package vorhanden
\usepackage{bbm}
\usepackage[ngerman]{babel}
\usepackage[T1]{fontenc}
\usepackage[latin9]{inputenc}
\usepackage{enumitem}
\usepackage{geometry}
\usepackage{graphicx}
\usepackage{epstopdf}
\usepackage{pgf,tikz}
\usepackage{url}
\usepackage{hyperref}
% TODO Package
\usepackage{todonotes}

\geometry{a4paper,tmargin=3cm,bmargin=3cm,lmargin=3cm,rmargin=3cm,headheight=0cm,headsep=1cm,footskip=1cm}

%NUMMERIERUNG

\numberwithin{equation}{section}

%UMGEBUNGEN:

\newtheorem{theorem}{Satz}[section]
\newtheorem{definition}[theorem]{Definition}
\newtheorem{corollary}[theorem]{Korollar}
\newtheorem{lemma}[theorem]{Lemma}
\newtheorem{proposition}[theorem]{Satz}
\newtheorem{conjecture}[theorem]{Vermutung}
\newtheorem{problem}[theorem]{Problem}
\newtheorem{remark}[theorem]{Bemerkung}
\newtheorem{example}[theorem]{Beispiel}
%
\newenvironment{proof}{\noindent{\bf Beweis:}}{\hfill{\hbox{\rule{2mm}{2mm}}}\par\medskip}
%
\newenvironment{lateproof}[1]{\noindent{\bf Beweis von #1:}}{\hfill{\hbox{\rule{2mm}{2mm}}}\par\medskip}
\newenvironment{subproof}{\noindent{\em Bew.:}}{\hfill{\hbox{\frame{\rule{0mm}{1.75mm}\rule{1.75mm}{0mm}}}}\par\medskip}

%SATZ�BERSCHRIFT

\newcommand{\head}[1]{{\sc #1}}

%FALLUNTERSCHEIDUNG

\newcommand{\ncase}[2]{\smallskip {\bf Fall #1~:} {\it #2.}}

%DEFINITIONEN

\newcommand{\DF}[1]{{\bf #1\/}}

%BESONDERE SCHRIFTARTEN:

\newcommand{\cG}{{\cal G}}
\newcommand{\cE}{{\cal EG}}

%GRAPHENPARAMETER

\newcommand{\De}{\Delta}
\newcommand{\om}{\omega}
\newcommand{\al}{\alpha}
\newcommand{\cn}{\chi}

%DATUM

\def\datum{26. September 2014}

% MENGEN

\newcommand{\N}{{\mathbb{N}}}
\newcommand{\Z}{{\mathbb{Z}}}
\newcommand{\R}{{\mathbb{R}}}
\newcommand{\C}{{\mathbb{C}}}
\newcommand{\Rnn}{{\mathbb{R}^{n\times n}}}

% SONSTIGES

\newcommand{\tr}{\operatorname{trace}}
\newcommand{\set}[2]{\{#1 \;|\; #2 \}}
\newcommand{\ems}{\varnothing}
\newcommand{\sm}{\setminus}
\newcommand{\la}{\langle}
\newcommand{\ra}{\rangle}

% Nummerierung in align* Umgebung mit \numberthis
\newcommand\numberthis{\addtocounter{equation}{1}\tag{\theequation}}

% LITERATUR :

\newcommand{\au}[1]{{\sc #1}\,,}     % author(s)
\newcommand{\ti}[1]{{\rm #1.}}   % article title ohne abstand
\newcommand{\bkti}[1]{{\em #1}}      % book title
%\newcommand{\jou}[1]{{\em #1,\,}}    % journal
\newcommand{\jou}[1]{{\em #1\,}}
\newcommand{\yr}[1]{{\rm (#1).\,}}   % year
\newcommand{\pp}[1]{{\rm pp. #1.}} % pages
%\newcommand{\vp}[1]{{\rm #1.}}  % volume and pages
\newcommand{\vp}[2]{{\bf #1} {#2.}}
\def\und{{\rm und \/}}
