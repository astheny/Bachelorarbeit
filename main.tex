\documentclass[a4paper,11pt]{article}

\input{Vorspann/Vorspann.tex}


\begin{document}

\renewcommand{\baselinestretch}{1.5}\normalsize

\title{
    \begin{figure}[htbp]
        \vspace{-48pt}
        \centering
        \includegraphics[height=36pt]{TU-Logo.eps}
    \end{figure}\vspace{-16pt}
    \Large Fakultät für Mathematik und Naturwissenschaften\\
    \vspace{5pt}
    \small Institut für Mathematik, Arbeitsgruppe Diskrete Mathematik und Algebra\\
    \vspace{100pt}
  \Huge {\bf Bachelorarbeit}\\
  \vspace{90pt}
    \huge {\bf Chromatische Zahl und Spektrum von Graphen}\\
    \vspace{100pt}
}

\author{
    \begin{tabular}{rl}
      vorgelegt von~:& Stefan Heyder\\
     Matrikelnummer~:& 49070\\
           Betreuer~:& Prof.~Dr. Michael Stiebitz
\end{tabular}
    \vspace{20pt}
}
\date{\datum}
\maketitle
\thispagestyle{empty}


\newpage
\tableofcontents
\thispagestyle{empty}
\newpage
\pagenumbering{arabic}


\include{Kapitel1/Kapitel1}
\include{Kapitel2/Kapitel2}
%\include{Kapitel3}



% Literaturverzeichnis
\newpage
\addcontentsline{toc}{section}{Literatur}
\bibliographystyle{plain}
\begin{thebibliography}{1}

\bibitem{Brooks1941}
\au{Brooks, R.L.} \yr{1941} \ti{On colouring the nodes of a network}
\jou{Proc. Cambride Phil. Soc.}  \vp{37}{194--197}

\bibitem{Diestel2010}
\au{Diestel, R.} \yr{2010} \bkti{Graphentheorie.} Springer.

\bibitem{Erdos1979}
\au{Erd\H{o}s, P.} \yr{1979}
\ti{Problems and results in graph theory and combinatorial analysis}
In: \bkti{Graph Theory and Related
Topics} (Proc. Conf., Univ. Waterloo, Waterloo, Ont., 1977), \pp{153--163}
Academic Press.

\end{thebibliography}




\newpage
\thispagestyle{empty}
\vspace*{15cm}
\begin{tabular}{lp{10cm}}
{Erklärung:} & {Hiermit versichere ich, dass ich diese Bachelorarbeit selbstständig verfasst und nur die angegebene Literatur verwendet habe. Die Arbeit wurde bisher keiner Prüfungsbehörde vorgelegt und auch noch nicht veröffentlicht.} \\
\\\\
\multicolumn{2}{l}
{Ilmenau, \datum \hspace{3.2cm} \rule{4cm}{0.4pt}}\\
\multicolumn{2}{l}
{\hspace{8.9cm} Stefan Heyder} \\
\end{tabular}




\end{document}
