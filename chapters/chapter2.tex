\section{Krauszzerlegungen und die Erd\H{o}s--Faber--Lov\'asz Vermutung}
\subsection{Die Erd\H{o}s--Faber--Lov\'asz Vermutung}
\label{sec:EFL-Vermutung}
Es bezeichne $\cE(n)$ die Klasse aller Graphen welche die Vereinigung von $n$ kantendisjunkten vollst"andigen Graphen der Ordnung $n$ sind. F"ur $G\in\cE(n)$ gilt also $G= \bigcup\limits_{i=1}^{n} G_i$, wobei $G_i \cong K_n$ und $|G_i \cap G_j| \leq 1$ f"ur alle $1\leq i,j \leq n$ mit $i\neq j$. 
\todo{Bilder, Beispiele}
\begin{conjecture}[Erd\H{o}s--Faber--Lov\'asz]
  Sei $G\in\cE(n)$. Dann ist $\chi(G) = n$.
  \label{con:efl}
\end{conjecture}

\begin{remark}
  \ref{con:efl} gilt offenbar f"ur $n = 1,2$. 
\end{remark}

\subsection{Krauszzerlegungen von Graphen}
\label{ssec:Krauszzerlegung}
\begin{definition}
  \label{def:Krauszzerlegung}
  Sei $G$ ein Graph. Eine Menge $\mathcal K$ von Untergraphen von $G$ hei"st \DF{Krauszzerlegung} von $G$, falls gilt:
  \begin{enumerate}[label=(\roman*)]
    \item Alle Graphen $K\in \mathcal{K}$ sind vollst"andige Graphen mit $|K| \geq 2$.
    \item Sind $K,K'$ zwei verschiedene Graphen aus $\mathcal{K}$, so sind sie kantendisjunkt (also $|K\cap K'| \leq 1$)
    \item $\mathcal K$ ist eine Zerlegung von $G$, also  $G=\bigcup\limits_{K\in \mathcal K}K$
  \end{enumerate}
  Desweiteren sei f"ur $v\in V(G)$ der \DF{Grad} von $v$ bez"uglich $\mathcal K$ definiert als $$d_G(v:\mathcal K) := |\{ K\in\mathcal K| v \in V(K)\}|$$ und der \DF{Minimalgrad} von $G$ bez"uglich $\mathcal K$ als $$\delta_G(\mathcal K) := \min\limits_{v\in V(G)}d_G(v:\mathcal K)$$ 
  F"ur $d \geq 1$ sei $\kappa_d(G)$ die kleinste positive Zahl $m$ derart, dass $G$ eine Krauszzerlegung $\mathcal K$ besitzt mit $|\mathcal K| = m$ und $\delta_G(\mathcal K) \geq d$. Existiert keine solche Zahl $m$, so setzen wir $\kappa_{d}(G) = \infty$.
\end{definition}
%% evtll: K1 dazu, dann delta + 1
\todo{Bilder, Beispiele}
\begin{lemma}
  Sei $G$ ein Graph. Dann ist $\kappa_d(G) < \infty$ genau dann, wenn $\delta(G) \geq d$.
  \label{lm:krauszexistenz}
\end{lemma}

\begin{proof}
  Wir zeigen zun"achst, dass $\delta(G) \geq d$, falls $\kappa_d(G) < \infty$. Seien $\kappa_d(G) <\infty$ und $v\in V(G)$ mit $d_{G}(v) = \delta(G)$. Dann existieren $d$ kantendisjunkte vollst"andige Untergraphen $H^{1}\dots H^{d}$ von $G$ mit $v\in V(H^{i})$ f"ur alle $1\leq i \leq d$. Da die $H^{i}$ kantendisjunkt sind und $d_{H^{i}}(v)\geq1$, gilt: 

  \[
    d \leq \sum\limits_{i=1}^{d} d_{H^{i}}(v) \leq d_G(v) = \delta(G). 
  \]
  Sei nun $\delta(G) \geq d$. Wir m"ussen zeigen, dass es eine Krauszzerlegung $\mathcal{K}$ gibt, mit $d_{G}(v:\mathcal{K}) \geq d$ f"ur alle $v\in V(G)$. Sei $E(G)= \{e_1,\dots, e_{m}\}$ eine Nummerierung der Kanten. Setze $H^{i}= G[e_i]$. Dann ist $\mathcal{K} = \{H^{1},\dots, H^{m}\}$ eine Krauszzerlegung von $G$ mit $d_{G}(v:\mathcal{K}) = d_{G}(v) \geq \delta(G) \geq d$ f"ur alle $v\in V(G)$. Also ist $\kappa_d(G)\leq m <\infty$.

\end{proof}
%Beispiele und einfache Ergebnisse
\begin{theorem}
  Die folgenden Aussagen sind "aquivalent:
  \begin{enumerate}[label=(\alph*)]
    \item F"ur alle $n\in\N$ und alle $G\in\cE(n)$ gilt $\chi(G) = n$.
    \item Ist $G$ ein Graph mit Minimalgrad mindestens $2$, so ist $\chi(G) \leq \kappa_{2}(G)$.
    \item Ist $H$ ein linearer Hypergraph so gilt $\chi'(H)\leq |H|$
  \end{enumerate}
  \label{thm:equivefl}
\end{theorem}

\begin{proof}
  Wir zeigen zun"achst, dass (b) aus (a) folgt. Sei $G$ ein Graph mit $\kappa_{2}(G) =m$. Dann existiert eine Krauszzerlegung $\mathcal{K} = \left\{ K_1,\dots K_m \right\}$ von $G$. Ist $m \geq n$
  \begin{align*}
    \chi(G) \leq n \leq m = \kappa_2 (G)
  \end{align*}
  Ist andererseits $m< n $, so ist $|K^{i}| \leq \omega (G)  \leq \kappa_2 (G) =  m$ (siehe \ref{cor:alphaomegakrausz}). Damit k"onnen wir f"ur $1\leq i \leq n$ jeden $K^{i}$ durch Hinzuf"ugen von Ecken und Kanten zu einem vollst"andigen Graphen von Grad $m$ aufbl"ahen. Den so enstehenden Graphen nennen wir $G'$. Offenbar ist $G$ ein Untergraph von $G'$ und $G'\in \cE(m)$. Damit gilt 
  \begin{align*}
    \chi(G) \leq \chi (G') = m = \kappa_{2}(G)
  \end{align*}
  Also folgt (b) aus (a).  

  Um zu zeigen, dass (a) aus (b) folgt, sei $G\in \cE(n)$. Dann ist $G$ die kantendisjunkte Vereinigung von $n$ vollst"andigen Graphen der Ordnung $n$. W"ahle $X= \left\{ v\in V(G) | d_G(v) \geq n \right\}$ und betrachte den von $X$ induzierten Untergraphen $H$ von $G$. Die Einschr"ankung der vollst"andigen Graphen, aus denen $G$ besteht, auf $H$ ergeben eine Krauszzerlegung $\mathcal{K}$ von $H$ (jeder induzierte Untergraph eines vollst"andigen Graphen ist wieder ein vollst"andiger
  Graph). Die Wahl von $X$ sichert, dass $\delta_G(\mathcal K) \geq 2$. Somit 
  \begin{align*}
    \chi(H) \leq \kappa_{2}(H) \leq \kappa_{2}(G) \leq n
  \end{align*}
  Also erhalten wir eine F"arbung von $H$ mit maximal $n$ Farben. Diese l"asst sich auf $G$ zu einer F"arbung mit maximal $n$ Farben erweitern, da wir nur Ecken entfernt haben, deren Grad kleiner als $n$ ist, welche folglich nur $n-1$ Nachbarn haben. 
  Damit ist die "Aquivalenz von (a) und (b) gezeigt. Es bleibt die "Aquivalenz von (b) und (c) zu zeigen. 

  Gelte 3. und sei $G$ ein Graph mit Minimalgrad gr"o{\ss}er oder gleich $2$. Seien weiterhin $\kappa_{2}(G) = m $ und $\mathcal{K}= \left\{ K^{1},\dots,K^{m} \right\}$ eine Krauszzerlegung von $G$. 
  F"ur $v\in V(G)$ definiere $e_v$ als die Menge aller $K\in \mathcal{K}$, welche $v$ enthalten. Sind dann $v$ und $w$ zwei unterschiedliche Ecken von $G$, so sind $e_v$ und $e_{w}$ unterschiedlich, da sonst Eigenschafft (ii) der Krauszzerlegung verletzt w"are. 
  Sei $H$ der Hypergraph mit Eckenmenge $\mathcal{K}$ und Kantenmenge $\left\{ e_v| v\in V(G) \right\}$. 
  Wir zeigen, dass $H$ linear ist. 
  Seien dazu $e_{v},e_{w} $ zwei unterschiedliche Kanten von $H$.
  Angenommen $|e_{v}\cap e_{w}| \geq 2$. Dann existieren mindestens zwei vollst"andige Graphen $K,K'$, welche sowohl $v$ als auch $w$ enthalten. Dies ist ein Widerspruch zu Eigenschaft (ii) der Krauszzerlegung. Folglich ist $|e_{v}\cap e_{w}| \leq 1$, also ist $H$ linear. Dann folgt aus der Vorraussetzung, dass $\chi'(H) \leq |H| = \kappa_{2}(G)$ 
  Somit finden wir eine F"arbung der Kanten von $H$ mit maximal $\kappa_{2}(G)$ Farben. Jede Kante von $H$ korrespondiert mit genau einer Ecke von $G$. Folglich erhalten wir eine F"arbung von $G$ mit maximal $\kappa_{2}(G)$ Farben, wenn wir die Ecken von $G$ mit der selben Farbe wie ihre korrespondierende Kante in $H$ f"arben. \todo{Format} Sonst exisiteren zwei Ecken $v,w$ von $G$ die benachbart sind und die selbe Farbe haben, 
  also auch $e_v$ und $e_w$. Da $v$ und $w$ benachbart sind, kommen sie beide im selben vollst"andigen Graph $K$ der Krauszzerlegung vor. Damit ist $|e_v\cap e_w| = 0$ und folglich kommen $v$ und $w$ nicht in den selben vollst"andigen Graphen vor. Widerspruch. Damit folgt
  \begin{align*}
    \chi(G) \leq \chi(H) \leq \kappa_{2}(G)
  \end{align*}

  Gelte 2. und sei $H$ ein (von der M"achtigkeit) minimaler linearer Hypergraph welcher 3. nicht erf"ullt. Ist dann $\delta(H) \leq 1$ so k"onnen wir eine Ecke aus $H$ entfernen und erhalten einen linearen Hypergraphen $H'$ kleinerer M"achtigkeit, welcher ebenfalls 3. nicht erf"ullt. Also ist $\delta(H) \geq 2$. Sei $G$ der Kantengraph von $H$ mit $V(G) = E(H)$ und $E(G) = \left\{ ee'|e\cap e' \neq \emptyset \right\}$. Definere f"ur $v\in V(H)$ die Menge $E_{H}(v)$ als die Menge aller
  Kanten von $H$, die mit $v$ inzident sind und $K^{v}$ als den durch $E_H(v)$ induzierten Untergraphen von $G$.
  Wir zeigen: $\mathcal K := \left\{ K^{v}| v \in V(H) \right\}$ ist eine Krauszzerlegung von $G$.  

  Seien $e,e'$ zwei verschiedene Kanten aus $E_{H}(v)$. Dann haben $e$ und $e'$ die Ecke $v$ gemeinsam. Folglich ist $e\cap e' \neq \emptyset$ und $ee'$ ist eine Kante in $G$. Au{\ss}erdem ist 
  \begin{align*}
    |K^{v}| = |E_H(v)| = d_{H}(v) \geq 2
  \end{align*}
  Damit erf"ullen alle $K^{v}$ Bedingung (i) von \ref{def:Krauszzerlegung}. Bedingung (ii) folgt, da $H$ ein linearer Hypergraph ist \todo{ausf"uhrlicher}. (iii) folgt, da jede Kante von $H$ mit mindestens einer Ecke von $H$ inzident ist.
\end{proof}
\subsection{Krauszzerlegungen und Eigenwerte}

\begin{theorem}
  \label{thm:KrauszEigenwerte}
  Sei $G$ ein Graph mit $V(G)=\{v_1,\dots,v_n\}$ und $\mathcal K=\{K_1,\dots,K_m\}$ eine Krauszzerlegung von $G$ mit $d_G(\mathcal K) \geq d \geq 2$ . Desweiteren sei $d_i := d_G(v:\mathcal K)$. 
  Wir k"onnen ohne Beschr"ankung der Allgemeinheit annehmen, dass $d_1\geq d_2 \geq \dots \geq d_n \geq d$.
  Dann gilt: 
  \begin{enumerate}[label=(\alph*)]
    \item $\lambda_i(G) \geq -d_{n-i+1}$ f"ur $i = 1, \dots , n$
    \item $\lambda_{m+1}(G) \leq -d$ falls $m < n$
  \end{enumerate}
\end{theorem}
\begin{proof}
  \begin{enumerate}[label=(\alph*)]
    \item Es sei $A$ die Adjazenzmatrix von $G$ und $D := \operatorname{diag}(d_1,\dots,d_n)$. Definiere $B\in\R^{n\times m}$ als die Inzidenzmatrix von $\mathcal K$, also $$B_{i,j} = \begin{cases}
        1 & v_i \in K_j \\ 0 & v_i \notin K_j
      \end{cases}$$ 
      Nun betrachten wir $M=BB^{T}$. Es gilt
      \[
        M_{i,j} = \sum\limits_{k=1}^{d}B_{i,k}B_{j,k}
      \]
      Seien zun"achst $i,j \in \{1,\dots,n\}$ mit $i\neq j$. Ist $B_{i,k} = 1$ und $B_{j,k} = 1$, so ist $v_i,v_j  \in K_k$, und somit (da $K_k$ ein vollst"andiger Graph ist) $v_iv_j\in E(G)$. Es k"onnen aber f"ur h"ochstens ein $k\in \{1,\dots,m\}$ $B_{i,k}$ und $B_{j,k}$ gleichzeitig $1$ seien, da nach \ref{def:Krauszzerlegung} die Graphen aus $\mathcal K$ kantendisjunkt sind. Also ist $M_{i,j}= 1 $ genau dann, wenn $v_iv_j\in E(G)$, genau dann wenn $A_{i,j} = 1$. Somit ist $M_{i,j}=A_{i,j}$\\
      Sei nun $i\in\{1,\dots,n\}$ beliebig. Wir betrachten $M_{i,i}$. Es gilt 
      \[
        M_{i,i} = \sum\limits_{k=1}^{d}B_{i,k}B_{i,k} = \sum\limits_{k=1}^{d} B_{i,k}
      \]
      $B_{i,k}=1$ genau dann, wenn $v_i \in K_k$. Folglich ist $M_{i,i}= d_G(v_i:\mathcal K)= d_i$. Damit gilt $M=A+D$. $M$ ist nach \ref{prop:psdmatrix} positiv semidefinit.
      Folglich ist $A- (-D)$ positiv semidefinit, und es folgt mit \ref{lem:evpsddif}, dass 
      \begin{equation*}
        \lambda_i(G) = \lambda_i(A) \geq \lambda_i(-D) =- -d_{n-i+1}
      \end{equation*}
    \item Sei $m<n$. Dann ist $\operatorname{rang}(M)= \operatorname{rang}(B) \leq m$. Also ist $\lambda_{m+1}(M) = 0$ und es folgt mit \ref{thm:weylineq} dass 
      \begin{align*}
        \lambda_{m+1}(A) + d \leq \lambda_{m+1}(A) + d_{n} \leq 0
      \end{align*}
      Durch Umstellen erhalten wir die gew"unschte Ungleichung.
  \end{enumerate}
\end{proof}
\begin{corollary}
  \label{cor:Korollar1}
  \todo{ $\delta(H) \geq 2 ?$ }
  Sei $H$ ein induzierter Untergraph von $G$ mit $p = \left| H\right| \leq \left|G\right| = n$.
  Ist $\lambda_p (H) > -d $ f"ur ein $q \leq p$ und $-2 \leq d \in \N$, so ist $\kappa_d(G) \geq q$.
\end{corollary}
\begin{proof}
  Angenommen $\kappa_d(G) < q$. Sei dann $\mathcal{K}$ eine Krauszzerlegung von $G$ die zu $d$ passt. Dann ist nach \ref{thm:Interlacing} $\lambda_{q}(G)\geq \lambda_{q}(H) > -d$. 
  Da $i < q \leq n$ gilt nach \ref{thm:KrauszEigenwerte} $\lambda_{i+1}\leq -d$ und somit $\lambda_{q}(G)\leq \lambda_{i+1} \leq -d$. Widerspruch.
\end{proof}
Es sei $G$ ein Graph. Der \DF{Kantengraph} $L(G)$ (engl. line graph) von $G$ ist definiert als der Graph mit Eckenmenge $E(G)$ und Kantenmenge $\{ee'|e\text{ und } e' \text{ besitzen eine gemeinsame Endecke}\}$.

\begin{corollary}
  \label{cor:LineGraphWald}
  Sei $\delta(G) \geq 2$ und $H$ ein induzierter Untergraph von $G$, welcher Kantengraph eines Waldes ist. 
  Dann ist $\kappa_{2}(H)\geq \left|H\right|$
\end{corollary}

\begin{proof}
  Da $H$ Kantengraph eines Waldes ist, folgt $\lambda_{min}(H) > -2$ (vgl. \cite[3.4.10]{zbMATH05625877}) 
  Nach \ref{cor:Korollar1} ist f"ur $q=\left|H\right|$ : $\lambda_q(H)=\lambda_{\text{min}}> -2$. Dann ist mit \ref{cor:Korollar1} $\kappa_{2}\left( H \right) > \left| H\right|$.
\end{proof}

\begin{corollary}[Klotz]
  $\kappa_{2}\left( K_n \right) \geq n$
\end{corollary}

\begin{proof}
  $K_n$ ist der Kantengraph von $K_{1,n}$. Nun folgt die Behauptung aus \ref{cor:LineGraphWald}.
\end{proof}
\begin{corollary}
  Ist $\delta\left( G \right) \geq 2$, so gilt $\omega\left( G \right)\leq \kappa_{2}\left( G \right)$ und $\alpha\left( G \right)\leq \kappa_{2}\left( G \right)$.
  \label{cor:alphaomegakrausz}
\end{corollary}

\begin{proof}
  Sei $p = \omega(G)$. Dann gilt nach $\lambda_{p}\left( G \right)\geq -1\geq -2$. Damit sind f"ur $d=2$ die Voraussetzungen von \ref{cor:Korollar1} erf"ullt, und es gilt folglich $\kappa_{2}\left( G \right)\geq p = \omega\left( G \right)$ 
  F"ur $q=\alpha\left( G) \right)$ gilt wieder nach $\lambda_{q}\left( G \right)\geq 0 \geq -2$. Damit folgt analog $\alpha\left( G \right) \leq \kappa_{2}\left( G \right)$.
\end{proof}
\begin{conjecture}[$C_{d}$]
  \label{con:MainConjecture}
  Ist $\chi\left( G \right) \geq k$, so ist $\lambda_k\left( G \right) > -d$.
  \todo{Cd richtig anzeigen}
\end{conjecture}

\begin{theorem}
  \label{thm:MainTheorem}
  Gelte \ref{con:MainConjecture}. Dann gilt :
  \begin{enumerate}[label=(\roman*)]
    \item Ist $\delta\left( G \right) \geq 2$ so ist $\chi\left( G \right)\leq \kappa_{d}\left( G \right)$
    \item  Ist $H$ ein linearer Hypergraph mit $\left|e\right| \geq d$, so ist $\chi'\left( H \right)\leq \left|H\right| $
  \end{enumerate}
\end{theorem}

\begin{proof}
  Sei $\chi(G) = k$. Dann ist nach \ref{con:MainConjecture} $\lambda_{k}\left( G \right) > -2$. Mit \ref{cor:Korollar1} folgt$\kappa_{d}\left( G \right) \geq k = \chi\left( G \right)$.
\end{proof}


\subsection{Schranken f"ur $\kappa_d(G)$}

Wir wollen nun einige Schranken f"ur $\kappa_{d}(G)$ angeben. 
\begin{lemma}
  Ist $\delta(G) \geq d$, so ist $\kappa_{d}(G) \leq |E(G)|$. 
\end{lemma}
\begin{proof}
  Dies folgt unmittelbar aus dem Beweis von \ref{lm:krauszexistenz}. 
\end{proof}

\begin{theorem}
  \begin{align*}
    \kappa_{d}(G) &\geq \frac{nd}{\lambda_{1}(G) +d} 
    %\lambda_n(A) \leq -d 
  \end{align*}
  \label{thm:kappaineq1}
\end{theorem}

\begin{proof}
  Ist $\kappa_{d}(G) = \infty$, so ist nichts zu zeigen. \\
  \ncase{1}{$\kappa_{d}(G) \geq n$} Da $\lambda_{1}(G) \geq 0$, gilt 
  \begin{align*}
    \lambda_{1}(G) + d &\geq d \\
    1 &\geq \frac{d}{\lambda_{1}(G) + d }\\
    \kappa_{d}(G) \geq n &\geq \frac{nd}{\lambda_{1}(G)+d}
  \end{align*}
  \ncase{2}{$\kappa_{d}(G) < n$}
  Wir verwenden die selbe Bezeichnung wie in \ref{thm:KrauszEigenwerte}. Dann ist $M=A+D$ mit $\operatorname{rang} (M) = m \leq \kappa_{d}(G) < n$ und folglich $\lambda_{m+1}(M) = \dots = \lambda_{n}(M) = 0$. 

  Mit \ref{thm:kyfanineq} folgt dann : 
  \begin{align*}
    \sum\limits_{i=1}^{n} \lambda_{i}(D) &=\sum\limits_{i=1}^{n} \lambda_{i}(A) +\sum\limits_{i=1}^{n}  \lambda_{i}(D) \\
    &=\sum\limits_{i=1}^{n} \lambda_{i}(M) =\sum\limits_{i=1}^{m} \lambda_{i}(M) \\
    &\leq \sum\limits_{i=1}^{m} \lambda_{i}(A) +\sum\limits_{i=1}^{m} \lambda_{i}(D)
  \end{align*}
  Daraus folgt 
  \begin{align*}
    (n-m) d \leq (n-m) \lambda_n(D) \leq \sum\limits_{i=m+1}^{n} \lambda_{i}(D) \leq\sum\limits_{i=1}^{m} \lambda_{i}(A) \leq m\lambda_{1}(A)
  \end{align*}
  Durch Umstellen erhalten wir die gew"unschte Ungleichung.
\end{proof}
