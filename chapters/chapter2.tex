\section{Die Erd\H{o}s--Faber--Lov\'asz Vermutung}
\label{sec:EFL-Vermutung}


\subsection{Krauszzerlegungen von Graphen}
\label{ssec:Krauszzerlegung}
\begin{definition}
    \label{def:Krauszzerlegung}
    Sei $G$ ein Graph. Eine Menge $\mathcal K$ von Untergraphen von $G$ hei"st \DF{Krauszzerlegung} von $G$, falls gilt:
    \begin{enumerate}[label=(\roman*)]
        \item $\forall K \in \mathcal K:$ $K$ ist ein vollst"andiger Graph mit $|K| \geq 2$
        \item $\forall K, K'\in \mathcal K$ mit $K\neq K'$ gilt $|K\cap K'| \leq 1$
        \item $G=\bigcup\limits_{K\in \mathcal K}K$
    \end{enumerate}
    Desweiteren sei f"ur $v\in V(G)$ der \DF{Grad} von $v$ bez"uglich $\mathcal K$ definiert als $$d_G(v:\mathcal K) := |\{ K\in\mathcal K| v \in V(K)\}|$$ und der \DF{Minimalgrad} von $G$ bez"uglich $\mathcal K$ als $$\delta_G(\mathcal K) := \min\limits_{v\in V(G)}d_G(v:\mathcal K)$$ 
    F"ur $d \geq 1$ sei $\kappa_d(G)$ die kleinste positive Zahl $m$ derart, dass $G$ eine Krauszzerlegung $\mathcal K$ besitzt mit $|\mathcal K| = m$ und $\delta_G(\mathcal K) \geq d$.
\end{definition}

\begin{lemma}
  Sei $G$ ein Graph. Dann ist $\kappa_d(G) < \infty$ genau dann, wenn $\delta(G) \geq d$.
  \label{lm:krauszexistenz}
\end{lemma}

\begin{proof}
  Seien $\kappa_d(G) \leq\infty$ und $v\in V(G)$ mit $d_{G}(v) = \delta(G)$. Dann existieren $d$ kantendisjunkte vollst"andige Untergraphen $H^{1}\dots H^{d}$ von $G$ mit $v\in V(H^{i})$ f"ur alle $1\leq i \leq d$. Da die $H^{i}$ kantendisjunkt sind und $d_H^{i}(v)\geq1$, gilt: 
  \[
    d \leq \sum\limits_{i=1}^{d} d_{H^{i}}(v) \leq d_G(v) = \delta(G). 
  \]
  Sei nun $\delta(G) \geq d$. Wir m"ussen zeigen, dass es eine Krauszzerlegung $\mathcal{K}$ gibt, mit $d_{G}(v:\mathcal{K}) \geq d$. Sei $E(G)= \{e_1,\dots, e_{m}\}$ eine Nummerierung der Kanten. Setze $H^{i}= G[e_i]$. Dann ist $\mathcal{K} = \{H^{1},\dots, H^{m}\}$ eine Krauszzerlegung von $G$ mit $d_{G}(v:\mathcal{K}) = d_{G}(v) \geq \delta(G) \geq d$ f"ur alle $v\in V(G)$. Also ist $\kappa_d(G)\leq m <\infty$.

\end{proof}
%Beispiele und einfache Ergebnisse
\subsection{Krauszzerlegungen und Eigenwerte}

\begin{theorem}
    \label{thm:KrauszEigenwerte}
    Sei $G$ ein Graph mit $V(G)=\{v_1,\dots,v_n\}$ und $\mathcal K=\{K_1,\dots,K_m\}$ eine Krauszzerlegung von $G$ mit $d_G(\mathcal K) \geq d \geq 2$ . Desweiteren sei $d_i := d_G(v:\mathcal K)$. 
    Wir k"onnen ohne Beschr"ankung der Allgemeinheit annehmen, dass $d_1\geq d_2 \geq \dots \geq d_n \geq d$.
    Dann gilt: 
    \begin{enumerate}[label=(\alph*)]
        \item $\lambda_i(G) \geq -d_{n-i+1}$ f"ur $i = 1, \dots , n$
        \item $\lambda_{m+1}(G) \leq -d$ falls $m < n$
    \end{enumerate}
\end{theorem}
\begin{proof}
    Es sei $A$ die Adjazenzmatrix von $G$ und $D := \operatorname{diag}(d_1,\dots,d_n)$. Definiere $B$ als die Inzidenzmatrix von $\mathcal K$, also $$B(i,j) = \begin{cases}
        1 & v_i \in K_j \\ 0 & v_i \notin K_j
    \end{cases}$$ 
    $M=BB^{T}$ ist positiv semidefinit, und es gilt $M = A+D$ \todo{Genauer}. Folglich ist $A- (-D)$ positiv semidefinit, und es folgt mit \ref{lem:evpsddif}, dass 
    \begin{equation*}
      \lambda_i(G) = \lambda_i(A) \geq \lambda_i(-D) =- -d_{n-i+1}
    \end{equation*}
  \end{proof}
\begin{corollary}
    \label{cor:Korollar1}
     \todo{ $\delta(H) \geq 2 ?$ }
    Sei $H$ ein induzierter Untergraph von $G$ mit $p = \left| H\right| \leq \left|G\right| = n$.
    Ist $\lambda_p (H) > -d $ f"ur ein $q \leq p$ und $-2 \leq d \in \N$, so ist $\kappa_d(G) \geq q$.
\end{corollary}
\begin{proof}
    Angenommen $\kappa_d(G) < q$. Sei dann $\mathcal{K}$ eine Krauszzerlegung von $G$ die zu $d$ passt. Dann ist nach \ref{thm:Interlacing} $\lambda_{q}(G)\geq \lambda_{q}(H) > -d$. 
    Da $i < q \leq n$ gilt nach \ref{thm:KrauszEigenwerte} $\lambda_{i+1}\leq -d$ und somit $\lambda_{q}(G)\leq \lambda_{i+1} \leq -d$. Widerspruch.
\end{proof}
Es sei $G$ ein Graph. Der \DF{Kantengraph} $L(G)$ (engl. line graph) von $G$ ist definiert als der Graph mit Eckenmenge $E(G)$ und Kantenmenge $\{ee'|e\text{ und } e' \text{besitzen eine gemeinsame Endecke}\}$.

\begin{corollary}
    Sei $\delta(G) \geq 2$ und $H$ ein induzierter Untergraph von $G$, welcher Kantengraph eines Waldes ist. 
    Dann ist $\kappa_{2}(H)\geq \left|H\right|$
\end{corollary}

\begin{proof}
  \todo{Satz Linegraphen Eigenwerte}
    Nach \ref{cor:Korollar1} ist f"ur $q=\left|H\right|$ : $\lambda_q(H)=\lambda_{\text{min}}> -2$. Dann ist mit \ref{cor:Korollar1} $\kappa_{2}\left( H \right) > \left| H\right|$.
\end{proof}

\begin{corollary}[Klotz]
    $\kappa_{2}\left( K_n \right) \geq n$
\end{corollary}

\begin{proof}
    $K_n$ ist der Kantengraph von $K_{1,n}$.
\end{proof}
\begin{corollary}
    Ist $\delta\left( G \right) \geq 2$, so gilt $\omega\left( G \right)\leq \kappa_{2}\left( G \right)$ und $\alpha\left( G \right)\leq \kappa_{2}\left( G \right)$.
\end{corollary}

\begin{proof}
    Sei $p = \omega(G)$. Dann gilt nach $\lambda_{p}\left( G \right)\geq -1\geq -2$. Damit sind f"ur $d=2$ die Voraussetzungen von \ref{cor:Korollar1} erf"ullt, und es gilt folglich $\kappa_{2}\left( G \right)\geq p = \omega\left( G \right)$ 
    F"ur $q=\alpha\left( G) \right)$ gilt wieder nach $\lambda_{q}\left( G \right)\geq 0 \geq -2$. Damit folgt analog $\alpha\left( G \right) \leq \kappa_{2}\left( G \right)$.
\end{proof}
\begin{conjecture}
    \label{con:MainConjecture}
    Ist $\chi\left( G \right) \geq k$, so ist $\lambda_k\left( G \right) > -d$.
\end{conjecture}

\begin{theorem}
    \label{thm:MainTheorem}
    Gelte \ref{con:MainConjecture}. Dann gilt :
    \begin{enumerate}[label=(\roman*)]
        \item Ist $\delta\left( G \right) \geq 2$ so ist $\chi\left( G \right)\leq \kappa_{d}\left( G \right)$
        \item  Ist $H$ ein linearer Hypergraph mit $\left|e\right| \geq d$, so ist $\chi'\left( H \right)\leq \left|H\right| $
    \end{enumerate}
\end{theorem}

\begin{proof}
    Sei $\chi(G) = k$. Dann ist nach \ref{con:MainConjecture} $\lambda_{k}\left( G \right) > -2$. Mit \ref{cor:Korollar1} folgt$\kappa_{d}\left( G \right) \geq k = \chi\left( G \right)$.
\end{proof}
