\section{Spektraleigenschaften von Graphen}

In diesem Kapitel wollen wir die in Abschnitt \ref{sss:ewgraph} behandelten Themen weiter vertiefen. Insbesondere werden wir einen Zusammenhang zwischen Krauszzerlegungen und den Eigenwerten eines Graphen herstellen. Da spezielle Krauszzerlegungen von Graphen (n"amlich diejenigen mit Minimalgrad $\geq 2$) in Zusammenhang mit Vermutung \ref{con:efl} stehen, werden wir hier eine alternative Herangehensweise an die Vermutung finden. 

\subsection{Krauszzerlegungen und Eigenwerte}
\begin{theorem}
  \label{thm:KrauszEigenwerte}
  Seien $G$ ein Graph mit $V(G)=\{v_1,\dots,v_n\}$ und $\mathcal K=\{K^1,\dots,K^p\}$ eine Krauszzerlegung von $G$ mit $d(\mathcal K) \geq d \geq 2$ . Desweiteren sei $d_i = d_{\mathcal{K}}(v_{i})$ f"ur $1\leq i \leq n$. 
  Dabei w"ahlen wir die Eckennummerierung so, dass $d_1 \geq \dots \geq d_{n}$ ist.
  Dann gelten folgende Aussagen : 
  \begin{enumerate}[label=\rm{(\alph*)}]
    \item $\lambda_i(G) \geq -d_{n-i+1}$ f"ur $i = 1, \dots , n$.
    \item $\lambda_{p+1}(G) \leq -d$ falls $p < n$ ist.
  \end{enumerate}
\end{theorem}
\begin{proof}
  Zun"achst zeigen wir (a). Es sei $A$ die Adjazenzmatrix von $G$ und $D := \operatorname{diag}(d_1,\dots,d_n)$. Definiere $B\in\R^{n\times m}$ als die Inzidenzmatrix von $\mathcal K$, also $$B_{ij} = \begin{cases}
    1 & v_i \in K^j \\ 0 & \text{falls }v_i \notin K^j
  \end{cases}$$ 
  Nun betrachten wir $M=BB^{T}$. Es gilt
  \[
    M_{ij} = \sum\limits_{k=1}^{d}B_{ik}B^{T}_{kj} = \sum\limits_{k=1}^{d}B_{ik}B{jk}
  \]
  Seien $i,j \in \{1,\dots,n\}$ mit $i\neq j$. Da $B$ die Inzidenzmatrix von $\mathcal{K}$ ist, gilt  $$ B_{ik} = 1 \text{ und } B_{jk} = 1 \Leftrightarrow v_i,v_j \in K^{k}.$$ Ist $v_iv_j \in E(G)$, so kommt die Kante $v_iv_j$ in genau einem $K\in \mathcal{K}$ vor, d.h. es gibt genau ein $k\in \left\{ 1,\dots,m \right\}$ f"ur das $B_{ik}$ und $B_{jk}$ gleich $1$ sind. Ist $v_iv_j\notin E(G)$, so kommt die Kante $v_iv_j$ auch nicht in einem der Graphen der Krauszzerlegung vor.
  Also ist f"ur alle $k\in \left\{ 1,\dots, m \right\}$ $B_{ik}B_{jk} = 0$. Folglich ist $M_{ij}=1$ genau dann, wenn $v_iv_j\in G$. Also ist $M_{ij} = A_{ij}$.
  
  Sei nun $i\in\{1,\dots,n\}$ beliebig. Wir betrachten $M_{ii}$. Es gilt 
  \[
    M_{ii} = \sum\limits_{k=1}^{d}B_{ik}B_{ik} = \sum\limits_{k=1}^{d} B_{ik}.
  \]
  $B_{ik}=1$ gilt genau dann, wenn $v_i \in K^k$. Folglich ist $M_{ii}= d_{\mathcal{K}}(v_i)= d_i$. Damit gilt $M=A+D$. $M=BB^{T}$ ist nach Satz \ref{prop:psdmatrix} positiv semidefinit.
  Folglich ist $A- (-D)$ positiv semidefinit, und es folgt mit Lemma \ref{lem:evpsddif}, dass 
  \begin{equation*}
    \lambda_i(G) = \lambda_i(A) \geq \lambda_i(-D) = -d_{n-i+1}
  \end{equation*}
  Damit ist (a) gezeigt.

  Nun zeigen wir (b). Sei $p<n$. Dann ist $\operatorname{rang}(M)= \operatorname{rang}(B) \leq p$. Also ist $\lambda_{p+1}(M) = 0$ und es folgt mit Satz \ref{thm:weylineq} dass 
  \begin{align*}
    \lambda_{p+1}(A) + d \leq \lambda_{p+1}(A) + d_{n} = \lambda_{p+1}(A) + \lambda_{n} (D) \leq \lambda_{p+1} (M) = 0
  \end{align*}
  Durch Umstellen erhalten wir die gew"unschte Ungleichung.
\end{proof}
\begin{corollary}
  \label{cor:Korollar1}
  Seien $G$ ein Graph und $H$ ein induzierter Untergraph von $G$. Desweiteren seien $q,d \in \N$ mit $q \leq |H|$ und $d \geq 2$.
  Ist $\lambda_{q}(H) \geq -d$, so ist $\kappa_{d}(G) > q$.
\end{corollary}
\begin{proof}
  Angenommen es gilt $p = \kappa_{d}(G) < q$. Dann gibt es  eine Krauszzerlegung $\mathcal{K}$ von $G$ mit $|\mathcal{K}| = p$ und $\delta(\mathcal{K}) \geq d$. Wegen Lemma \ref{lem:InterlacingGraphen} gilt dann $\lambda_{q}(G) \geq \lambda_{q}(H) > -d $. Andererseits folgt aus Satz \ref{thm:KrauszEigenwerte} dass $\lambda_{q}(G) \leq \lambda_{p+1} \leq -d $, ein Widerspruch. 
\end{proof}

\begin{corollary}
  \label{cor:LineGraphWald}
  Seien $\delta(G) \geq 2$ und $H$ ein induzierter Untergraph von $G$. Ist $H$ Kantengraph eines Waldes, so gilt 
  $\kappa_{2}(G)\geq \left|H\right|$.
\end{corollary}

\begin{proof}
  Sei $q = |H|$. Da $H$ Kantengraph eines Waldes ist, folgt $\lambda_{q}(H) > -2$ aus Korollar \ref{cor:linegraphwald}.
  Dann ist mit Korollar \ref{cor:Korollar1} $\kappa_{2}\left( G \right) \geq \left| H\right|$.
\end{proof}

\begin{corollary}[Klotz]
  $\kappa_{2}\left( K_n \right) \geq n$
\end{corollary}

\begin{proof}
  $K_n$ ist der Kantengraph von $K_{1,n}$. Nun folgt die Behauptung aus Korollar \ref{cor:LineGraphWald}.
\end{proof}
\begin{corollary}
  Ist $\delta\left( G \right) \geq 2$, so gilt $\omega\left( G \right)\leq \kappa_{2}\left( G \right)$ und $\alpha\left( G \right)\leq \kappa_{2}\left( G \right)$.
  \label{cor:alphaomegakrausz}
\end{corollary}

\begin{proof}
  Sei $p = \omega(G)$. Dann gilt nach Korollar \ref{cor:alphaomegaEigenwerte} $\lambda_{p}\left( G \right)\geq -1 > -2$. Damit sind f"ur $d=2$ die Voraussetzungen von Korollar \ref{cor:Korollar1} erf"ullt, und es gilt folglich $\kappa_{2}\left( G \right)\geq p = \omega\left( G \right)$ .
  F"ur $q=\alpha\left( G \right)$ gilt mit Korollar \ref{cor:alphaomegaEigenwerte} $\lambda_{q}\left( G \right)\geq 0 > -2$. Damit folgt $\alpha\left( G \right) \leq \kappa_{2}\left( G \right)$.
\end{proof}

Wir wollen nun einen Menge von Graphenparametern definieren. Seien dazu $d\geq 1$ und $G$ ein Graph. Wir bezeichnen mit $\xi_{d}$ die Anzahl aller Eigenwerte von $G$, welche echt gr"o{\ss}er als $-d$ sind. Damit ist also 
$$\xi_{d}(G) = \max \set{i}{\lambda_i > -d}.$$
Dieser ist ein Graphenparamter, da nach Lemma \ref{lem:GraphEigenwerte} isomorphe Graphen das selbe Spektrum besitzen, und somit $\xi_d $ zwei isomorphen Graphen die selbe reelle Zahl zuordnet. 
Nach Lemma \ref{lem:InterlacingGraphen} ist $\xi_{d}$ f"ur alle $d\geq 1$ ein monotoner Graphenparamter, d.h. ist $H$ ein induzierter Untergraph von $G$, so gilt $\xi_d(H) \leq \xi_d(G)$.
Um die Erd\H{o}s-Faber-Lov\'asz Vermutung zu beweisen, reicht es also aus zu zeigen, dass jeder Graph $$\chi(G) \leq \xi_{2}(G)$$ erf"ullt. 

Eine M"oglichkeit dies zu zeigen, w"are zu beweisen, dass $\xi_2$ ein Szekeres-Wilf-Paramter ist. Die Eigenschaft (S1) ist nach den obigen "Uberlgungen erf"ullt. Berechnungen in Maple zeigen jedoch, dass Eigenschaft (S2) nicht immer erf"ullt ist. \todo{ausf"uhrlicher? mit Beispiel?}
\begin{theorem}
  \label{thm:MainTheorem}
  Existiert ein $d\in \N$, sodass f"ur alle  Graphen $G$ $$\chi(G) \leq \xi_{d}(G)$$ gilt, so gelten folgende Aussagen:
  \begin{enumerate}[label=\rm{(\alph*)}]
    \item F"ur alle Graphen $G$ gilt $\chi(G) \leq \kappa_d (G)$.
    \item  Ist $H$ ein linearer Hypergraph mit $\left|e\right| \geq d$ f"ur alle $e\in E(H)$, so ist $\chi'\left( H \right)\leq \left|H\right| $
  \end{enumerate}
\end{theorem}

\begin{proof}
  Wir zeigen zun"achst (a). Sei $G$ ein beliebiger Graph mit $\chi(G) = k$. Nach Voraussetzung ist dann $\chi(G) \leq \xi_{d}(G)$, d.h. $\lambda_{k}\left( G \right) > -d$. Mit Korollar \ref{cor:Korollar1} folgt $\kappa_{d}\left( G \right) \geq k = \chi\left( G \right)$. 
  Damit ist (a) gezeigt. 

  Wir zeigen nun (b) durch Widerspruch. Angenommen die Behauptung gilt nicht. Dann gibt es einen linearen Hypergraphen minimaler Ordnung mit $|e| \geq d$ f"ur alle $e\in E(H)$ , f"ur welchen $\chi'(H) > H$. 
  Wir machen eine Fallunterscheidung bez"uglich dem kleinsten Eckengrad.
  
  \ncase{1} Es existiert eine Kante $e$ von $H$ vom Grad kleiner als $d$. Da $|e| \geq d$ ist und $H$ ein linearer Hypergraph ist, gibt es eine Ecke $v$ von $H$ welche nur in $e$ vorkommt. Sei $H'= H-e$. Dann ist $|H'| < |H|$. Dieser l"asst sich auf Grund der Wahl von $H$ mit $\chi'(H') \leq |H'|$ Farben f"arben. Diese F"arbung k"onnen wir zu einer F"arbung von $H$ mit h"ochstens $|H'|+1 = |H|$ Farben erweitern, indem wir $e$ mit einer neuen Farbe f"arben. Dann gilt 
  $$\chi'(H) \leq \chi'(H) +1  \leq |H'| +1 = |H| .$$ 
  Ein Widerspruch zur Wahl von $H$. 

  \ncase{2} Alle Kanten von $H$ haben mindestens den Grad $d$. 
  Sei $G=L(H)$ der Kantengraph von $H$. Dann ist $\delta(G) \geq d$. Sei $K^{v} = G[E_{H}(v)]$ f"ur alle $v\in V(G)$. Sei $\mathcal{K}=\left\{ K^{v} | v \in V(H) \right\}$. Dann ist $\mathcal{K}$ eine Krauszzerlegung von $G$. mit $\delta_{G}(\mathcal{K}) \geq d$. Damit gilt
  \begin{equation*}
    \chi'(H) = \chi(G) \leq \kappa_{d}(G) \leq |\mathcal{K}| = |H|.
  \end{equation*}
  Wobei die erste Ungleichung wegen (a) gilt. 
\end{proof}


\subsection{Schranken f"ur $\kappa_d(G)$}

Wir wollen nun einige Schranken f"ur $\kappa_{d}(G)$ angeben. 
\begin{lemma}
  Ist $\delta(G) \geq d$, so ist $\kappa_{d}(G) \leq |E(G)|$. 
\end{lemma}
\begin{proof}
  Dies folgt unmittelbar aus dem Beweis von Lemma \ref{lm:krauszexistenz}. 
\end{proof}

\begin{theorem}
  Sei $G$ ein Graph der Ordnung $n$ und $d\in \N$. Dann gilt:
  \begin{align*}
    \kappa_{d}(G) &\geq \frac{nd}{\lambda_{1}(G) +d} 
    %\lambda_n(A) \leq -d 
  \end{align*}
  \label{thm:kappaineq1}
\end{theorem}

\begin{proof}
  Ist $\kappa_{d}(G) = \infty$, so ist nichts zu zeigen. \\
  \ncase{1}{$\kappa_{d}(G) \geq n$} Da $\lambda_{1}(G) \geq 0$, gilt 
  \begin{align*}
    \lambda_{1}(G) + d &\geq d \\
    1 &\geq \frac{d}{\lambda_{1}(G) + d }\\
    \kappa_{d}(G) \geq n &\geq \frac{nd}{\lambda_{1}(G)+d}
  \end{align*}
  \ncase{2}{$\kappa_{d}(G) < n$}
  Sei $\mathcal{K}$ eine Krauszzerlegung von $G$ mit $|\mathcal{K}| = \kappa_{d}(G)$ und $\delta_{G}(\mathcal{K}) \geq d$. Seien $d_{i} = d_{\mathcal{K}}(v)$. Wir k"onnen annehmen, dass die $d_{i}$ fallend geordnet sind. Sei $B\in \R^{n\times p}$ die Adjanzenzmatrix von $\mathcal{K}$ und $M = BB^{T} = A+D$, wobei $A= A(G)$ und $D = \operatorname{diag}(d_{1},\dots,d_n)$.
  Dann ist $M$ positiv semidefinit und $\operatorname{rang} (M) \leq p = \kappa_{d}(G) < n $. Deswegen ist $\lambda_{p+1}(M) = \dots \lambda_{n}(M) = 0$. 
  Mit Satz \ref{thm:kyfanineq} folgt dann : 
  \begin{align*}
    \sum\limits_{i=1}^{n} \lambda_{i}(D) &=\sum\limits_{i=1}^{n} \lambda_{i}(A) +\sum\limits_{i=1}^{n}  \lambda_{i}(D) \\
    &=\sum\limits_{i=1}^{n} \lambda_{i}(M) =\sum\limits_{i=1}^{p} \lambda_{i}(M) \\
    &\leq \sum\limits_{i=1}^{p} \lambda_{i}(A) +\sum\limits_{i=1}^{p} \lambda_{i}(D)
  \end{align*}
  Daraus folgt 
  \begin{align*}
    (n-p) d \leq (n-p) \lambda_n(D) \leq \sum\limits_{i=m+1}^{n} \lambda_{i}(D) \leq\sum\limits_{i=1}^{p} \lambda_{i}(A) \leq p\lambda_{1}(A)
  \end{align*}
  Durch Umstellen nach $p$ erhalten wir die gew"unschte Ungleichung.
\end{proof}

\subsection{Chromatische Zahl und Eigenwerte}
Es ist nicht viel "uber den Zusammenhang der chromatischen Zahl eines Graphen, und seinen Eigenwerte bekannt. Wir wollen hier auf zwei S"atze verweisen, die Schranken f"ur die chromatische Zahl eines Graphen in Abh"angigkeit des gr"o{\ss}ten bzw. kleinsten Eigenwerts angeben. Eine Absch"atzung nach oben  gibt Wilf in \cite{wilf1967eigenvalues} an. Man beachte, dass diese eine Verst"arkung von Satz \ref{thm:brooks} ist, da der gr"o{\ss}te Eigenwert eines Graphen duch den Maximalgrad
beschr"ankt ist.

\begin{theorem}
  Ist $G$ ein Graph, so gilt: 
  $$\chi(G) \leq \lambda_{1}(G) +1.$$
  Gleichheit tritt nur auf, falls $G$ ein vollst"ander Graph oder ein ungerader Kreis ist.
  \label{thm:wilfev}
\end{theorem}

Eine untere Schranke findet sich in \cite{Hoffmanbounds} (man beachte hierbei, dass $\lambda_{min}(G)$ negativ ist):

\begin{theorem}
  Ist $G$ ein Graph, so gilt:
  $$\chi(G) \geq 1 - \frac{\lambda_{max}(G)}{\lambda_{min}(G)}$$
  \label{thm:Hoffmanev}
\end{theorem}

\subsection{Graphen mit $\chi \leq \xi_{2}$}

Gelten die Vorraussetzungen von Satz \ref{thm:MainTheorem} f"ur $d=2$, so folgt die Erd\H{o}s--Faber--Lov\'asz Vermutung auf Grund von Satz \ref{thm:equivefl}. Im Folgenden wollen wir f"ur einige Graphenklassen folgende Vermutung "uberpr"ufen.
\begin{conjecture}
  Ist $G$ ein Graph so gilt $\chi(G) \leq \xi_{2}(G)$
  \label{con:maincon}
\end{conjecture}

\begin{remark}
  Es reicht Vermutung \ref{con:maincon} f"ur $k$-kritische Graphen zu zeigen. 
\end{remark}

\begin{proof}
  Gelte Vermutung \ref{con:maincon} f"ur kritische Graphen.
  Sei $G$ ein Graph mit $\chi(G) = k$. Dann enth"alt $G$ einen $k$-kritischen Untergraphen $H$. F"ur diesen gilt nach Vorraussetzung $$k= \chi(H) \leq \xi_{2}(H).$$ Folglich ist $\lambda_{k}(H) > -2$. Nun folgt mit Satz \ref{lem:InterlacingGraphen}:
  \begin{equation*}
    \lambda_{k}(G) \geq \lambda_{k}(H) > -2.
  \end{equation*}
  Und deswegen $\chi(G) = k \leq \xi_{2}(G)$.
  Damit gilt Vermutung \ref{con:maincon} auch f"ur $G$.
\end{proof}

Seien $v,r$ zwei nat"urliche Zahlen mit $v\geq 2r-1$. Der \DF{Kneser Graph} $K_{v:r}$ geht auf Kneser \cite{Kneser55} zur"uck und ist der Graph mit Eckenmenge $$V(K_{v:r}) = \left\{ X \subset \left\{ 1,\dots v \right\} | |X| = r \right\}$$ und Kantenmenge 
 $$E(K_{v:r}) = \left\{ XY| X,Y \in V(K_{v:r}), X \cap Y = \emptyset \right\}.$$ 
 $K_{v:1}$ ist ein vollst"andiger Graph der Ordnung $v$ und $K_{2r-1:r}$ besitzt keine Kanten. Kneser gibt in \cite{Kneser55} eine obere Schranke f"ur die chromatische Zahl der Kneser Graphen an. Sp"ater zeigte Lov\'asz \todo{Referenz}, dass diese Schranke immer angenommen wird.
 \begin{theorem}
   Ist $v\geq 2r-1$, dann gilt
   $$\chi(K_{v:r}) \leq v-2r+2$$
   \label{thm:kneserfarbung}
 \end{theorem}
 Der folgende Satz ist als Satz von Erd\H{o}s-Ko-Rado \cite{ErdosKoRado61} bekannt, und gibt die Unabh"angigkeitszahl eines Kneser Graphen an.
 \begin{theorem}
   Ist $v\geq 2r$ und $r \geq 2$, so gilt: 
   $$\alpha(K_{v:r})= \binom{v-1}{r-1}.$$
   \label{thm:ErdosKoRado}
 \end{theorem}

\begin{proposition}
  Seien $v\geq 2r-1$, $r\geq 1$ und sei $G= K_{v:r}$ ein Kneser Graph. Dann gilt $\chi(G) \leq \xi_{2}(G)$. 
\end{proposition}

\begin{proof}
  Sei $\chi(G) = k$. Wir fixieren $v\in \N$ und machen eine Fallunterscheidung bez"uglich $r$. 

  \ncase{1} {$r=1$} Dann ist $G=K_{v:1}$ isomorph zu $K_v$. Die Eigenwerte des $K_v$ sind alle gr"o{\ss}er als $-2$, insbesondere also auch $\lambda_{k}(G)$. Folglich ist $\chi(G) \leq \xi_{2}(G)$. 

  \ncase{2} {$v > 2r \geq 4 $} Sei $p = \alpha(G)$.  Dann ist nach Satz \ref{thm:ErdosKoRado} $ p = \binom{v-1}{r-1}$ und folglich $ p \geq v-1$. Nach Satz \ref{thm:kneserfarbung} ist $\chi(G) = v-2r+2 < v-2$. Folglich ist $p > \chi(G)$. Mit  Korollar \ref{cor:alphaomegaEigenwerte} gilt dann \begin{equation*}
    \lambda_{k}(G) \geq \lambda_{p}(G) \geq 0 > -2.
  \end{equation*}
  Folglich gilt $\chi(G) \leq \xi_{2}(G)$.
  \ncase{3} {$ v = 2r $} Die Ecken von $G$ sind alle $r$-elementingen Teilmengen von $\left\{ 1,\dots,v \right\}$. Da $v=2r$ ist f"ur ein $w\in V(G)$ die einzige benachbarte Ecke ihr Komplement in $\left\{ 1,\dots,v \right\}$. Also sind die Komponenten von $G$ alle isomorph zu $K_{2}$. Dann ist $\chi(G) = 2$ und $\omega(G) = 2$. Aus Korollar \ref{cor:alphaomegaEigenwerte} folgt dann, dass $\lambda_{2}(G) \geq -1 > -2$ ist. Insbesondere ist $\chi(G) \leq \xi(G)$. 

  \ncase{4}{$v=2r+1$} Dann ist $|E(G)| = 0$, und folglich ist $G$ ein kantenloser Graph, welcher nur den Eigenwert $0>-2$ besitzt. Insbesondere ist also auch $\lambda_{k}(G) = 0 > -2$. Und deswegen gilt $\chi(G) \leq \xi_{2}(G)$.
\end{proof}

\begin{proposition}
  Sei $G$ ein perfekter Graph. Dann gilt:
  $$\chi(G) \leq \xi_{2}(G)$$
\end{proposition}

\begin{proof}
  Da $G$ ein perfekter Graph ist, gilt $k = \chi(G) = \omega(G)$. Also besitzt $G$ einen vollst"andigen Graphen der Ordnung $k$ als induzierten Untergraphen. Nach \ref{cor:alphaomegaEigenwerte} ist $\lambda_{k}(G) > -2$. Und damit $\chi(G) \leq \xi_{2}(G)$.
\end{proof}

\begin{proposition}
  Vermutung \ref{con:maincon} gilt f"ur planare Graphen welche mindestens $7$ Ecken besitzen.
\end{proposition}

\begin{proof}
  Sei $G$ ein planarer Graph mit $|G| \geq 7$.
  Wir zeigen, dass $\lambda_4(G) > -2$ gilt. Da alle planaren Graphen eine chromatische Zahl kleiner gleich $4$ haben, folgt dann die Behauptung.

  Nehmen wir daf"ur an, dass $\lambda_{4}(G) < -2$ gilt.
  Aus Lemma \ref{rem:evGraph} (i) folgt dann:
  \begin{equation*}
    \sum\limits_{i=1}^3 \lambda_{i}(G) = -\sum\limits_{i=4}^{n}\lambda_{i}(G) \geq 2 (n-3) = 2n-6
  \end{equation*}
  Mit Lemma \ref{rem:evGraph} (ii) folgt au{\ss}erdem:
  \begin{align*}
    \sum\limits_{i=1}^3 \lambda_{i}(G)^{2} &= 2m -\sum\limits_{i=4}^{n}\lambda_{i}(G)^{2} \\
    &\leq 2(3n-6) -4(n-3) = 2n
  \end{align*}
  Die Ungleichung folgt dabei aus der Absch"artzung der Kanten f"ur planare Graphen. Aus der Absch"atzung der $2$ und $1$ Norm folgt nun:
  \begin{align*}
    2n &\geq \lambda_{1}^{2} +\lambda_{2}^{2} +\lambda_{3}^{2}  \\ &\geq \frac{1}{3} (|\lambda_{1}| + |\lambda_2| + |\lambda_3|)^{2} \\
    &\geq \frac{1}{3}(\lambda_{1}+ \lambda_2 + \lambda_3)^{2}\\ &\geq \frac{1}{3}(2n-6)^{2}
  \end{align*}
  L"osen der entstehenden quadratischen Gleichung ergibt $n\leq 6$, ein Widerspruch.
\end{proof} 
\subsection{H\'ajos und Ore Summe}
In diesem Abschnitt wollen wir die Klasse der H\'ajos-konstruierbaren Graphen und die Klasse der Ore-konstruierbaren Graphen untersuchen. 

Sei $G$ ein Graph und $X$ eine nichtleere Menge von Ecken von $G$. Wir erhalten einen neuen Graphen $G'$ aus $G$ durch \DF{Identifizierung}  von $X$ mit einer Ecke $x$, falls $G'$ aus $G-X$ durch hinzuf"ugen der neuen Ecke $x$ entsteht, welche mit allen Nachbarn von $X$ in $G-X$ benachbart ist. 

Seien $G_1, G_2$ zwei nichtleere disjunkte Graphen und $x_1y_1\in E(G_1)$ sowie $x_2y_2 \in E(G_2)$ zwei beliebige Kanten. Die \DF{Haj\'os Summe} \cite{Hajos61} von $G_1$ und $G_2$ (bez"uglich $x_1y_1$ und $x_2y_2$) ist derjenige Graph, der entsteht wenn wir $G_1$ und $G_2$ vereinigen, die Kanten $x_1y_1$ und $x_2y_2$ entfernen, die Kante $y_1y_2$ hinzuf"ugen und  die Ecken $x_1$ und $x_2$ identifizieren. Wir schreiben dann auch $G= \text{H\'ajos}(G_1,x_1y_1,G_2,x_1y_2)$ oder
kurz $G= \text{H\'ajos}(G_1,G_2)$ falls die Wahl der Kanten beliebig ist.

F"ur $k\in \N$ hei{\ss}t ein Graph $G$ \DF{H\'ajos-$k$-konstruierbar}, falls $G$ entweder ein vollst"andiger Graph der Ordnung $k$ ist, falls $G$ durch die Haj\'os Summe zweier Haj\'os $k$-konstruierbarer Graphen entsteht, oder falls $G$ durch die Identifizierung zweier nicht benachbarten Ecken eines Haj\'os $k$-konstruierbaren Graphen entsteht. 

\todo{ein Bild?}
Es ist bekannt, dass die H\'ajos Summe zweier $k$-chromatischer Graphen wieder $k$-chromatisch ist \todo{quelle?}. 

\begin{theorem}
  Seien $k\in\N$ und $d\in\N$. Seien weiterhin $G_1$ und $G_2$ Graphen mit $k\leq \xi_{d}(G)$. Dann gilt f"ur $G= \text{H\'ajos}(G_1,G_2)$:
  $$\xi_{d}(G) \geq 2k-4$$
  \label{thm:hajoseigenwerte}
\end{theorem}

\begin{proof}
  Seien $x_1y_1\in E(G_1)$ und $x_2y_2\in E(G_2)$ die Kanten welche bei der H\'ajos Summe verwendet werden. Wir setzen $$H_i = G[V(G_i)\setminus\{x_i,y_i\}] $$ f"ur $i=1,2$. Dann ist f"ur $i=1,2$ der Graph $H_i$ sowohl ein induzierter Untergraph von $G_i$ als auch von $G$. 
  Desweiteren gilt $|H_i| = |G_i|-2$ f"ur $i=1,2$. Also folgt aus Lemma \ref{lem:InterlacingGraphen}, dass $$\lambda_{k-2}(H_i) \geq \lambda_{k}(G_i) > -d$$ f"ur $i=1,2$ richtig ist. Wir betrachten nun den Graphen $H$, welcher die disjunkte Vereinigung von $H_1$ und $H_2$ ist. Dann ist $H$ ebenfalls ein induzierter Untergraph von $G$. 
  Das Spektrum von $H$ ist nach Beispiel \ref{ex:disjointunion} die Vereinigung der Spektra von $H_1$ und $H_2$. Also gilt $\xi_{d}(H) \geq 2(k-2) = 2k-4$, da sowohl $H_1$ als auch $H_2$ mindestens $k-2$ Eigenwerte besitzen, die gr"o{\ss}er als $-d$ sind. 
  Da $\xi_d$ ein monotoner Graphenparamter ist, gilt auch $\xi_{d}(G) \geq 2k-4$.
\end{proof}

Eine weitere, der H\'ajos Summe "ahnliche Konstruktion ist die \DF{Ore Summe} \cite{Ore67}. Seien $G_1$ und $G_2$ zwei disjunkte Graphen und $x_1y_1$ eine Kante von $G_1$ und $x_2y_2$ eine Kante von $G_2$. Desweiteren seien $v_1,v_2,\dots,v_t$ unterschiedliche Ecken von $G_1\setminus x_1$ und $u_1,u_2,\dots,u_r$ unterschiedliche Ecken von $G_2\setminus x_2$. Wir konstruieren nun die Ore Summe von $G_1$ und $G_2$ aus der H\'ajos Summe
$\text{H\'ajos}(G_1,x_1y_1, G_2, x_2y_2)$, indem wir zus"atzlich die Kanten $v_1u_1,v_2u_2,\dots,v_ru_r$ hinzuf"ugen. Der so entstandene Graph ist dann die Ore Summe von $G_1$ und $G_2$. 

F"ur $k\in\N$ hei{\ss}t ein Graph $G$ \DF{Ore $k$-konstruierbar}, falls $G$ entweder ein vollst"andiger Graph der Ordnung $k$ ist, oder die Ore Summe zweier disjunkter Ore $k$-konstruierbaren Graphen ist. 
