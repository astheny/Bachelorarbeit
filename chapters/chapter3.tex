\section{Spektraleigenschaften von Graphen}

In diesem Kapitel wollen wir die in Abschnitt \ref{sss:ewgraph} behandelten Themen weiter vertiefen. Insbesondere werden wir einen Zusammenhang zwischen Krauszzerlegungen und den Eigenwerten eines Graphen herstellen. Da spezielle Krauszzerlegungen von Graphen (n"amlich diejenigen mit Minimalgrad $\geq 2$) in Zusammenhang mit Vermutung \ref{con:efl} stehen, werden wir hier eine alternative Herangehensweise an die Vermutung finden. 

\subsection{Krauszzerlegungen und Eigenwerte}
\begin{theorem}
  \label{thm:KrauszEigenwerte}
  Seien $G$ ein Graph mit $V(G)=\{v_1,\dots,v_n\}$ und $\mathcal K=\{K^1,\dots,K^p\}$ eine Krauszzerlegung von $G$ mit $d(\mathcal K) \geq d \geq 2$ . Desweiteren sei $d_i = d_{\mathcal{K}}(v_{i})$ f"ur $1\leq i \leq n$. 
  Dabei w"ahlen wir die Eckennummerierung so, dass $d_1 \geq \dots \geq d_{n}$ ist.
  Dann gelten folgende Aussagen : 
  \begin{enumerate}[label=\rm{(\alph*)}]
    \item $\lambda_i(G) \geq -d_{n-i+1}$ f"ur $i = 1, \dots , n$.
    \item $\lambda_{p+1}(G) \leq -d$ falls $p < n$ ist.
  \end{enumerate}
\end{theorem}
\begin{proof}
  Zun"achst zeigen wir (a). Es sei $A$ die Adjazenzmatrix von $G$ und $D := \operatorname{diag}(d_1,\dots,d_n)$. Definiere $B\in\R^{n\times m}$ als die Inzidenzmatrix von $\mathcal K$, also $$B_{ij} = \begin{cases}
    1 & v_i \in K^j \\ 0 & \text{falls }v_i \notin K^j
  \end{cases}$$ 
  Nun betrachten wir $M=BB^{T}$. Es gilt
  \[
    M_{ij} = \sum\limits_{k=1}^{d}B_{ik}B^{T}_{kj} = \sum\limits_{k=1}^{d}B_{ik}B{jk}
  \]
  Seien $i,j \in \{1,\dots,n\}$ mit $i\neq j$. Da $B$ die Inzidenzmatrix von $\mathcal{K}$ ist, gilt  $$ B_{ik} = 1 \text{ und } B_{jk} = 1 \Leftrightarrow v_i,v_j \in K^{k}.$$ Ist $v_iv_j \in E(G)$, so kommt die Kante $v_iv_j$ in genau einem $K\in \mathcal{K}$ vor, d.h. es gibt genau ein $k\in \left\{ 1,\dots,m \right\}$ f"ur das $B_{ik}$ und $B_{jk}$ gleich $1$ sind. Ist $v_iv_j\notin E(G)$, so kommt die Kante $v_iv_j$ auch nicht in einem der Graphen der Krauszzerlegung vor.
  Also ist f"ur alle $k\in \left\{ 1,\dots, m \right\}$ $B_{ik}B_{jk} = 0$. Folglich ist $M_{ij}=1$ genau dann, wenn $v_iv_j\in G$. Also ist $M_{ij} = A_{ij}$.
  
  Sei nun $i\in\{1,\dots,n\}$ beliebig. Wir betrachten $M_{ii}$. Es gilt 
  \[
    M_{ii} = \sum\limits_{k=1}^{d}B_{ik}B_{ik} = \sum\limits_{k=1}^{d} B_{ik}.
  \]
  $B_{ik}=1$ gilt genau dann, wenn $v_i \in K^k$. Folglich ist $M_{ii}= d_{\mathcal{K}}(v_i)= d_i$. Damit gilt $M=A+D$. $M=BB^{T}$ ist nach Satz \ref{prop:psdmatrix} positiv semidefinit.
  Folglich ist $A- (-D)$ positiv semidefinit, und es folgt mit Lemma \ref{lem:evpsddif}, dass 
  \begin{equation*}
    \lambda_i(G) = \lambda_i(A) \geq \lambda_i(-D) = -d_{n-i+1}
  \end{equation*}
  Damit ist (a) gezeigt.

  Nun zeigen wir (b). Sei $p<n$. Dann ist $\operatorname{rang}(M)= \operatorname{rang}(B) \leq p$. Also ist $\lambda_{p+1}(M) = 0$ und es folgt mit Satz \ref{thm:weylineq} dass 
  \begin{align*}
    \lambda_{p+1}(A) + d \leq \lambda_{p+1}(A) + d_{n} = \lambda_{p+1}(A) + \lambda_{n} (D) \leq \lambda_{p+1} (M) = 0
  \end{align*}
  Durch Umstellen erhalten wir die gew"unschte Ungleichung.
\end{proof}
\begin{corollary}
  \label{cor:Korollar1}
  Seien $G$ ein Graph und $H$ ein induzierter Untergraph von $G$. Desweiteren seien $q,d \in \N$ mit $q \leq |H|$ und $d \geq 2$.
  Ist $\lambda_{q}(H) \geq -d$, so ist $\kappa_{d}(G) > q$.
\end{corollary}
\begin{proof}
  Angenommen es gilt $p = \kappa_{d}(G) < q$. Dann gibt es  eine Krauszzerlegung $\mathcal{K}$ von $G$ mit $|\mathcal{K}| = p$ und $\delta(\mathcal{K}) \geq d$. Wegen Lemma \ref{lem:InterlacingGraphen} gilt dann $\lambda_{q}(G) \geq \lambda_{q}(H) > -d $. Andererseits folgt aus Satz \ref{thm:KrauszEigenwerte} dass $\lambda_{q}(G) \leq \lambda_{p+1} \leq -d $, ein Widerspruch. 
\end{proof}

\begin{corollary}
  \label{cor:LineGraphWald}
  Seien $\delta(G) \geq 2$ und $H$ ein induzierter Untergraph von $G$. Ist $H$ Kantengraph eines Waldes, so gilt 
  $\kappa_{2}(G)\geq \left|H\right|$.
\end{corollary}

\begin{proof}
  Sei $q = |H|$. Da $H$ Kantengraph eines Waldes ist, folgt $\lambda_{q}(H) > -2$ aus Korollar \ref{cor:linegraphwald}.
  Dann ist mit Korollar \ref{cor:Korollar1} $\kappa_{2}\left( G \right) \geq \left| H\right|$.
\end{proof}

\begin{corollary}[Klotz]
  $\kappa_{2}\left( K_n \right) \geq n$
\end{corollary}

\begin{proof}
  $K_n$ ist der Kantengraph von $K_{1,n}$. Nun folgt die Behauptung aus Korollar \ref{cor:LineGraphWald}.
\end{proof}
\begin{corollary}
  Ist $\delta\left( G \right) \geq 2$, so gilt $\omega\left( G \right)\leq \kappa_{2}\left( G \right)$ und $\alpha\left( G \right)\leq \kappa_{2}\left( G \right)$.
  \label{cor:alphaomegakrausz}
\end{corollary}

\begin{proof}
  Sei $p = \omega(G)$. Dann gilt nach Korollar \ref{cor:alphaomegaEigenwerte} $\lambda_{p}\left( G \right)\geq -1 > -2$. Damit sind f"ur $d=2$ die Voraussetzungen von Korollar \ref{cor:Korollar1} erf"ullt, und es gilt folglich $\kappa_{2}\left( G \right)\geq p = \omega\left( G \right)$ .
  F"ur $q=\alpha\left( G \right)$ gilt mit Korollar \ref{cor:alphaomegaEigenwerte} $\lambda_{q}\left( G \right)\geq 0 > -2$. Damit folgt $\alpha\left( G \right) \leq \kappa_{2}\left( G \right)$.
\end{proof}

\begin{theorem}
  \label{thm:MainTheorem}
  Existiert ein $d\in \N$, sodass f"ur alle  Graphen $G$ mit $\chi(G) = k $ gilt $ \lambda_{k}(G) > -d$. Dann gelten folgende Aussagen:
  \begin{enumerate}[label=\rm{(\alph*)}]
    \item F"ur alle Graphen $G$ gilt $\chi(G) \leq \kappa_d (G)$.
    \item  Ist $H$ ein linearer Hypergraph mit $\left|e\right| \geq d$ f"ur alle $e\in E(H)$, so ist $\chi'\left( H \right)\leq \left|H\right| $
  \end{enumerate}
\end{theorem}

\begin{proof}
  Wir zeigen zun"achst (a). Sei $G$ ein beliebiger Graph mit $\chi(G) = k$. Nach Voraussetzung ist dann $\lambda_{k}\left( G \right) > -d$. Mit Korollar \ref{cor:Korollar1} folgt $\kappa_{d}\left( G \right) \geq k = \chi\left( G \right)$. 
  Damit ist (a) gezeigt. 

  Wir zeigen nun (b) durch Widerspruch. Angenommen die Behauptung gilt nicht. Dann gibt es einen linearen Hypergraphen minimaler Ordnung mit $|e| \geq d$ f"ur alle $e\in E(H)$ , f"ur welchen $\chi'(H) > H$. 
  Wir machen eine Fallunterscheidung bez"uglich dem kleinsten Eckengrad.
  
  \ncase{1} Es existiert eine Kante $e$ von $H$ vom Grad kleiner als $d$. Da $|e| \geq d$ ist und $H$ ein linearer Hypergraph ist, gibt es eine Ecke $v$ von $H$ welche nur in $e$ vorkommt. Sei $H'= H-e$. Dann ist $|H'| < |H|$. Dieser l"asst sich auf Grund der Wahl von $H$ mit $\chi'(H') \leq |H'|$ Farben f"arben. Diese F"arbung k"onnen wir zu einer F"arbung von $H$ mit h"ochstens $|H'|+1 = |H|$ Farben erweitern, indem wir $e$ mit einer neuen Farbe f"arben. Dann gilt 
  $$\chi'(H) \leq \chi'(H) +1  \leq |H'| +1 = |H| .$$ 
  Ein Widerspruch zur Wahl von $H$. 

  \ncase{2} Alle Kanten von $H$ haben mindestens den Grad $d$. 
  Sei $G=L(H)$ der Kantengraph von $H$. Dann ist $\delta(G) \geq d$. Sei $K^{v} = G[E_{H}(v)]$ f"ur alle $v\in V(G)$. Sei $\mathcal{K}=\left\{ K^{v} | v \in V(H) \right\}$. Dann ist $\mathcal{K}$ eine Krauszzerlegung von $G$. mit $\delta_{G}(\mathcal{K}) \geq d$. Damit gilt
  \begin{equation*}
    \chi'(H) = \chi(G) \leq \kappa_{d}(G) \leq |\mathcal{K}| = |H|.
  \end{equation*}
  Wobei die erste Ungleichung wegen (a) gilt. 
\end{proof}


\subsection{Schranken f"ur $\kappa_d(G)$}

Wir wollen nun einige Schranken f"ur $\kappa_{d}(G)$ angeben. 
\begin{lemma}
  Ist $\delta(G) \geq d$, so ist $\kappa_{d}(G) \leq |E(G)|$. 
\end{lemma}
\begin{proof}
  Dies folgt unmittelbar aus dem Beweis von Lemma \ref{lm:krauszexistenz}. 
\end{proof}

\begin{theorem}
  Sei $G$ ein Graph der Ordnung $n$ und $d\in \N$. Dann gilt:
  \begin{align*}
    \kappa_{d}(G) &\geq \frac{nd}{\lambda_{1}(G) +d} 
    %\lambda_n(A) \leq -d 
  \end{align*}
  \label{thm:kappaineq1}
\end{theorem}

\begin{proof}
  Ist $\kappa_{d}(G) = \infty$, so ist nichts zu zeigen. \\
  \ncase{1}{$\kappa_{d}(G) \geq n$} Da $\lambda_{1}(G) \geq 0$, gilt 
  \begin{align*}
    \lambda_{1}(G) + d &\geq d \\
    1 &\geq \frac{d}{\lambda_{1}(G) + d }\\
    \kappa_{d}(G) \geq n &\geq \frac{nd}{\lambda_{1}(G)+d}
  \end{align*}
  \ncase{2}{$\kappa_{d}(G) < n$}
  Sei $\mathcal{K}$ eine Krauszzerlegung von $G$ mit $|\mathcal{K}| = \kappa_{d}(G)$ und $\delta_{G}(\mathcal{K}) \geq d$. Seien $d_{i} = d_{\mathcal{K}}(v)$. Wir k"onnen annehmen, dass die $d_{i}$ fallend geordnet sind. Sei $B\in \R^{n\times p}$ die Adjanzenzmatrix von $\mathcal{K}$ und $M = BB^{T} = A+D$, wobei $A= A(G)$ und $D = \operatorname{diag}(d_{1},\dots,d_n)$.
  Dann ist $M$ positiv semidefinit und $\operatorname{rang} (M) \leq p = \kappa_{d}(G) < n $. Deswegen ist $\lambda_{p+1}(M) = \dots \lambda_{n}(M) = 0$. 
  Mit Satz \ref{thm:kyfanineq} folgt dann : 
  \begin{align*}
    \sum\limits_{i=1}^{n} \lambda_{i}(D) &=\sum\limits_{i=1}^{n} \lambda_{i}(A) +\sum\limits_{i=1}^{n}  \lambda_{i}(D) \\
    &=\sum\limits_{i=1}^{n} \lambda_{i}(M) =\sum\limits_{i=1}^{p} \lambda_{i}(M) \\
    &\leq \sum\limits_{i=1}^{p} \lambda_{i}(A) +\sum\limits_{i=1}^{p} \lambda_{i}(D)
  \end{align*}
  Daraus folgt 
  \begin{align*}
    (n-p) d \leq (n-p) \lambda_n(D) \leq \sum\limits_{i=m+1}^{n} \lambda_{i}(D) \leq\sum\limits_{i=1}^{p} \lambda_{i}(A) \leq p\lambda_{1}(A)
  \end{align*}
  Durch Umstellen nach $p$ erhalten wir die gew"unschte Ungleichung.
\end{proof}

\subsection{Chromatische Zahl und Eigenwerte}
Es ist nicht viel "uber den Zusammenhang der chromatischen Zahl eines Graphen, und seinen Eigenwerte bekannt. Wir wollen hier auf zwei S"atze verweisen, die Schranken f"ur die chromatische Zahl eines Graphen in Abh"angigkeit des gr"o{\ss}ten bzw. kleinsten Eigenwerts angeben. Eine Absch"atzung nach oben  gibt Wilf in \cite{wilf1967eigenvalues} an. Man beachte, dass diese eine Verst"arkung von Satz \ref{thm:brooks} ist, da der gr"o{\ss}te Eigenwert eines Graphen duch den Maximalgrad
beschr"ankt ist.

\begin{theorem}
  Ist $G$ ein Graph, so gilt: 
  $$\chi(G) \leq \lambda_{1}(G) +1.$$
  Gleichheit tritt nur auf, falls $G$ ein vollst"ander Graph oder ein ungerader Kreis ist.
  \label{thm:wilfev}
\end{theorem}

Eine untere Schranke findet sich in \cite{Hoffmanbounds} (man beachte hierbei, dass $\lambda_{min}(G)$ negativ ist):

\begin{theorem}
  Ist $G$ ein Graph, so gilt:
  $$\chi(G) \geq 1 - \frac{\lambda_{max}(G)}{\lambda_{min}(G)}$$
  \label{thm:Hoffmanev}
\end{theorem}

\subsection{$k$-chromatische Graphen mit $\lambda_{k} > -2$}

Gelten die Vorraussetzungen von Satz \ref{thm:MainTheorem} f"ur $d=2$, so folgt die Erd\H{o}s--Faber--Lov\'asz Vermutung auf Grund von Satz \ref{thm:equivefl}. Im Folgenden wollen wir f"ur einige Graphenklassen folgende Vermutung "uberpr"ufen.
\begin{conjecture}
  Ist $G$ ein Graph mit $\chi(G) = k$, dann gilt $\lambda_{k}(G) > -2$. 
  \label{con:maincon}
\end{conjecture}

\begin{remark}
  Es reicht Vermutung \ref{con:maincon} f"ur $k$-kritische Graphen zu zeigen. 
\end{remark}

\begin{proof}
  Sei $G$ ein Graph mit $\chi(G) = k$. Dann enth"alt $G$ einen $k$-kritischen Untergraphen $H$. F"ur diesen gilt (nach Vermutung) $\lambda_{k}(H) > -2$. Nun folgt mit Satz \ref{lem:InterlacingGraphen}:
  \begin{equation*}
    \lambda_{k}(G) \geq \lambda_{k}(H) > -2
  \end{equation*}
  Damit gilt Vermutung \ref{con:maincon} auch f"ur $G$.
\end{proof}

Seien $v,r$ zwei nat"urliche Zahlen mit $v\geq r$. Der \DF{Kneser Graph} $K_{v:r}$ geht auf Kneser \cite{Kneser55} zur"uck und ist der Graph mit Eckenmenge $$V(K_{v:r}) = \left\{ X \subset \left\{ 1,\dots v \right\} | |X| = r \right\}$$ und Kantenmenge 
 $$E(K_{v:r}) = \left\{ XY| X,Y \in V(K_{v:r}), X \cap Y = \emptyset \right\}.$$ 
 
Es wurde gezeigt \todo{Referenz}, dass $\alpha(K_{v:r}) = \binom{v-1}{r-1}$ (falls $v >2r$) und $\chi(K_{v:r}) = v-2r+2$. Das, und  Korollar \ref{cor:alphaomegaEigenwerte} erlaubt uns Vermutung \ref{con:maincon} f"ur alle Kneser Graphen zu beweisen. 

\begin{proposition}
  Seien $k,v\in \N$ mit $k>v$ und sei $G= K_{v:r}$ ein Kneser-Graph mit $\chi(G) = k$. Dann gilt $\lambda_{k}(G) > -2$
\end{proposition}

\begin{proof}
  Wir machen eine Fallunterscheidung bez"uglich $r$. 

  \ncase{1} {$r=1$} Dann ist $G=K_{v:1}$ isomorph zu $K_v$, da alle Ecken von $G$ einelementige Teilmengen von $\left\{ 1,\dots, v \right\}$ sind, und diese alle miteinander disjunkt sind. Die Eigenwerte des $K_v$ sind alle gr"o{\ss}er als $-2$. 

  \ncase{2} {$v > 2r \geq 4 $} Sei $p = \alpha(G)$.  Dann ist $ p = \binom{v-1}{r-1}$ und folglich $ p \geq v-1$. Andererseits ist $\chi(G) = v-2r+2 < v-2$. Folglich ist $p > \chi(G)$. Mit  Korollar \ref{cor:alphaomegaEigenwerte} gilt dann \begin{equation*}
    \lambda_{k}(G) \geq \lambda_{p}(G) \geq 0 > -2
  \end{equation*}

  \ncase{3} {$2r = v $} Die Ecken von $G$ sind alle $r$-elementingen Teilmengen von $\left\{ 1,\dots,v \right\}$. Da $v=2r$ ist f"ur ein $w\in V(G)$ die einzige benachbarte Ecke ihr Komplement in $\left\{ 1,\dots,v \right\}$. Also sind die Komponenten von $G$ alle isomorph zu $K_{2}$. Dann ist $\chi(G) = 2$ und $\omega(G) = 2$. Aus Korollar \ref{cor:alphaomegaEigenwerte} folgt dann, dass $\lambda_{2}(G) \geq -1 > -2$ ist.

  \ncase{4}{$2r > v$} Dann ist $|E(G)| = 0$, da je zwei Ecken nichtleeren Schnitt haben, und folglich ist $G$ ein kantenloser Graph, welcher nur den Eigenwert $0>-2$ besitzt. Insbesondere ist also auch $\lambda_{k}(G) > -2$.
\end{proof}

\begin{proposition}
  Sei $G$ ein perfekter Graph. Dann gilt f"ur $k=\chi(G)$:
  $$\lambda_{k}(G) > -2$$
\end{proposition}

\begin{proof}
  Da $G$ ein perfekter Graph ist, gilt $k = \chi(G) = \omega(G)$. Also besitzt $G$ einen vollst"andigen Graphen der Ordnung $k$ als induzierten Untergraphen. Nach \ref{cor:alphaomegaEigenwerte} ist $\lambda_{k}(G) > -2$.
\end{proof}

\begin{proposition}
  Vermutung \ref{con:maincon} gilt f"ur planare Graphen.
\end{proposition}

\begin{proof}
  Sei $G$ ein planarer Graph. Wir machen eine Fallunterscheidung nach $n = |G|$. 
  \ncase{1}{$n\leq6$}
  \todo{}

  \parindent 0pt \ncase{2}{$n \geq 7$}
  Wir zeigen, dass $\lambda_4(G) > -2$ gilt. Da alle planaren Graphen eine chromatische Zahl kleiner gleich $4$ haben, folgt dann die Behauptung.

  Nehmen wir daf"ur an, dass $\lambda_{4}(G) < -2$ gilt.
  Aus Lemma \ref{rem:evGraph} (i) folgt dann:
  \begin{equation*}
    \sum\limits_{i=1}^3 \lambda_{i}(G) = -\sum\limits_{i=4}^{n}\lambda_{i}(G) \geq 2 (n-3) = 2n-6
  \end{equation*}
  Mit Lemma \ref{rem:evGraph} (ii) folgt au{\ss}erdem:
  \begin{align*}
    \sum\limits_{i=1}^3 \lambda_{i}(G)^{2} &= 2m -\sum\limits_{i=4}^{n}\lambda_{i}(G)^{2} \\
    &\leq 2(3n-6) -4(n-3) = 2n
  \end{align*}
  Die Ungleichung folgt dabei aus der Absch"artzung der Kanten f"ur planare Graphen. Aus der Absch"atzung der $2$ und $1$ Norm folgt nun:
  \begin{align*}
    2n &\geq \lambda_{1}^{2} +\lambda_{2}^{2} +\lambda_{3}^{2}  \\ &\geq \frac{1}{3} (|\lambda_{1}| + |\lambda_2| + |\lambda_3|)^{2} \\
    &\geq \frac{1}{3}(\lambda_{1}+ \lambda_2 + \lambda_3)^{2}\\ &\geq \frac{1}{3}(2n-6)^{2}
  \end{align*}
  L"osen der entstehenden quadratischen Gleichung ergibt $n\leq 6$, ein Widerspruch.
\end{proof} 

