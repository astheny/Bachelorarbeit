\section{Einf"uhrung}
\label{sec:Einf"uhrung}
Gegenstand dieser Bachelorarbeit ist der Zusammenhang zwischen den Eigenwerten, der chromatischen Zahl und den Krauszzerlegungen eines Graphen. Die daf"ur ben"otigten Grundlagen werden wir in Kapitel 1 erarbeiten. 
\subsection{Graphen und Hypergraphen}
\label{ssec:graphenundhypergraphen}
Die in dieser Arbeit betrachteten Graphen und Hypergraphen sind endlich und haben weder Mehrfachkanten noch Schlingen. Bei den Bezeichnungen richten wir uns im Wesentlichen nach dem Buch von Diestel beziehungsweise dem Buch von Berge. \todo{Referenzen} 
Mit $\N$ bezeichnen wir die Menge der positiven ganzen Zahlen und setzen $\N_{0} = \N \cup \left\{ 0 \right\}$. F"ur eine Menge $V$ sei die Menge $2^{V}$ die Potenzmenge von $V$ und $[V]^{p}$ mit $p\in\N_0$ die Menge der $p$-elementigen Teilmengen von $V$. 

Ein \DF{Hypergraph} $H$ ist ein Tupel von zwei Menge, $V(H)$ und $E(H)$. Dabei ist $V(H)$ endlich und $E(H)$ eine Teilmenge von $2^{V(H)}$ mit $|e| \geq 2$ f"ur alle $e\in E(H)$. Die Menge $V(H)$ hei{\ss}t dann \DF{Eckenmenge} von $H$ und ihre Elemente hei{\ss}en \DF{Ecken} von $H$. Die Menge $E(H)$ hei{\ss}t \DF{Kantenmenge} und ihre Elemente hei{\ss}en \DF{Kanten}. Ein Hypergraph hei{\ss}t \DF{linear}, falls zwei verschieden Kanten h"ochstens eine Ecke gemeinsam haben.

Sei $H$ ein Hypergraph. Die \DF{Ordnung} von $H$ ist die Anzahl der Ecken von $H$, geschrieben $|H|$. Eine Kante $e$ hei{\ss}t \DF{Hyperkante}, falls $|e| \geq 3$ und sonst \DF{gew"ohnliche Kante}. F"ur eine gew"ohnliche Kante $e=\left\{ u,v \right\}$ schreiben wir auch kurz $e=uv$ oder $e=vu$. 
Ist $E(H) \subseteq [V]^p$, so nennen wir $H$ \DF{$p$-uniform}. Ein \DF{Graph} ist ein $2$-uniformer Hypergraph, also ein Hypergraph in dem jede Kante gew"ohnlich ist. Eine Ecke $v$ ist \DF{inzident} mit einer Kante $e$, falls $v\in e$ gilt. F"ur eine Ecke $v$ von $H$ sei $E_{H}(v) = \left\{ e\in E(H) | v\in e \right\}$. Der \DF{Grad} einer Ecke $v$ ist $d_{H}(v) = |E_H(v)|$. 
Der \DF{Minimalgrad} (\DF{Maximalgrad}) sei definiert als der kleinste (gr"o{\ss}te) Grad einer Ecke von $H$ und wird mit $\delta(H)$ ($\Delta(H)$) bezeichnet. Ist $\delta(H) = \Delta(H) = r$, so hei{\ss}t $H$ $r$-regul"ar. 

Ein \DF{Unterhypergraph} von $H$ ist ein Hypergraph $H'$ mit $V(H') \subseteq V(H)$ und $E(H') \subseteq E(H)$. Wir schreiben dann $H' \subseteq H$. Gilt $H' \neq H$, so ist $H'$ ein \DF{echter Unterhypergraph}. Gibt es eine Menge $X \subseteq V(H)$ mit $V(H') = X$ und $E(H') = \left\{ e\in E(H) | e \subset X \right\}$, so ist $H'$ ein \DF{induzierter Hypergraph} und wir schreiben $H'= H[X]$ beziehungsweise $H' \unlhd H$. 

Ist $H$ ein Hypergraph und $X\subseteq V(H)$, so bezeichne $H-X = H[V(H) \setminus X]$. Ist $X=\left\{ v \right\}$, so schreiben wir daf"ur auch $H-v$ statt $H-X$. Ist $F\subseteq 2^{V(H)}$ eine Menge, so sei $H-F$ der Hypergraph mit Eckenmenge $V(H)$ und Kantenmenge $E(H) \setminus F$ und $H+F$ der Hypergraph mit Eckenemenge $V(H)$ und Kantenmenge $E(H) \cup F$. Ist $F= \left\{ e \right\}$ so schreiben wir $H\pm e$ anstatt $H\pm  \left\{ e \right\}$. 

Eine Menge von Ecken $X\subseteq V(H)$ hei{\ss}t \DF{unabh"angige Menge} von $H$, falls $E(H[X]) = \emptyset$ gilt, beziehungsweise \DF{Clique}, falls $H[X]$ alle gew"ohnlichen Kanten von $[X]^{2}$ enth"alt. Die \DF{Unabh"angigkeitszahl} $\alpha(H)$ ist die Ordnung der gr"o{\ss}ten unabh"anigegen Menge von $H$. Die \DF{Cliquenzahl} $\omega(H)$ ist die Ordnung der gr"o{\ss}ten Clique von $H$. 

Sind $H,H'$ zwei Hypergraphen, so hei�t eine Abbildung $\varphi : V(H) \to V(H')$ \DF{Isomorphismus} zwischen $H$ und $H'$, falls f"ur alle Teilmengen $\left\{ v_1,v_2,\dots v_n \right\}$ von Ecken von $V(H)$ gilt: $$ \left\{ v_1,v_2,\dots v_n \right\} \text{ ist Kante von $H$} \Leftrightarrow \left\{ \phi(v_1),\phi(v_2),\dots \phi(v_n) \right\} \text{ ist Kante von $H'$}.$$ Zwei Hypergraphen $H,H'$ hei�en \DF{isomorph}, falls es einen Isomorphismus zwischen
$H$ und $H'$ gibt. 

Ein Graph $G$ hei{\ss}t \DF{vollst"andiger Graph}, falls $E(G) = [V(G)]^{2}$ gilt. Ist $G$ ein vollst"andiger Graph der Ordnung $n$, so schreiben wir auch $G=K_n$. Man beachte hierbei, dass alle vollst"andigen Graphen der Ordnung $n$ isomorph sind. In diesem Sinne bezeichnen wir mit $C_n$ den \DF{Kreis} der Ordnung $n$, mit $P_n$ den \DF{Weg} der Ordnung $n$ und mit $O_n$ den \DF{kantenlosen Graphen} der Ordnung $n$ (d.h. das Komplement von $K_n$). 

Damit ist $\omega(H)$ die gr"o{\ss}te Zahl, sodass $H$ einen vollst"andigen Graphen der Ordnung $n$ als Untergraphen enth"alt und $\alpha(H)$ die gr"o{\ss}te Zahl $n$, sodass $H$ den kantenlosen Graphen der Ordnung $n$ als induzierten Untergraphen enth"alt. 

\subsection{F"arbungen von Graphen und Hypergraphen}

Das \DF{F\"arbungsproblem} f\"ur Graphen ist ein klassischen Problem aus der Graphentheorie mit vielf\"altigen Anwendungen in der kombinatorischen Optimierung und anderen Teilgebieten der Mathematik. Beim  F\"arbungsproblem besteht die Aufgabe darin, die Ecken eines Graphen $G$ so zu f\"arben, dass durch eine Kante verbundene Ecken verschiedenen Farben erhalten. Dabei sollen nat\"urlich m\"oglichst wenige Farben verwendet werden.

Sei $C$ eine endliche Menge. Eine Abbildung $f:V(G) \to C$ hei{\ss}t \DF{F"arbung} von $G$, falls f"ur alle Kanten $vw$ von $G$ gilt: $f(v) \neq f(w)$. Ist $|C| = k \in \N$, so hei{\ss}t $f$ \DF{$k$-F"arbung}. Die kleinste nat"urliche Zahl $k$, f"ur die $G$ eine $k$-F"arbung besitzt, bezeichnen wir mit $\chi(G)$, der \DF{chromatischen Zahl} von $G$. 

Die Bestimmung der chromatischen Zahl eines Graphen ist ein {\sf NP}-schweres Optimierungsproblem, wie im Jahre 1972 von Karp \cite{karp} gezeigt wurde. Sei $f$ eine F"arbung von $G$ und $H$ ein Untergraph von $G$. Dann ist $f_{|V(H)}$ eine F"arbung von $H$. Folglich ist die chromatische Zahl ein monotoner Graphenparameter, d.h. $$ H\subseteq G \Rightarrow \chi(H) \leq \chi(G).$$

Eine Abbildung $f :V(G) \to C$ ist eine F"arbung von $G$, genau dann wenn f"ur alle $c\in C$ das Urbild $f^{-1}(c)$ eine unabh"angige Menge in $G$ ist (d.h. keine zwei Ecken von $f^{-1}(c)$ sind durch eine Kante von $G$ verbunden). Diese Urbilder nennen wir \DF{Farbklassen}. Offensichtlich sind die Farbklassen disjunkt. 
Folglich haben Farbklassen h"ochstens $\alpha (G)$ Ecken. Daraus folgt, dass jede $k$-F"arbung von $G$ $|G| \leq k \alpha(G)$ erf"ullt, und deswegen auch $|G| \leq \chi(G) \alpha(G)$ gilt. 

Da jede Ecke eine unabh"angige Menge ist, gilt $\chi(G) \leq |G|$. Damit gilt 
$$\chi(G) \geq |G|  \Leftrightarrow chi(G) = |G| \Leftrightarrow \alpha(G) \leq 1 \Leftrightarrow G \text{ ist ein vollst"andiger Graph}$$
Insbesondere gilt somit f"ur $n\in \N$ : $\chi(K_n)  = n $. Da $\chi $ ein monotoner Graphenparameter ist, ist also $$\omega(G) \leq \chi(G).$$

Die chromatische Zahl des Graphen $G$ ist die kleinste Zahl $k$, derart, dass sich $G$ in $k$ viele unabh"angige Mengen unterteilen l"asst. Deswegen gilt $\chi(G) = 0$ nur, falls $V(G) = \emptyset$, also $G=\emptyset$ der leere Graph ist. Au�erdem ist $\chi(G) \leq 1$ genau dann, wenn $G$ keine Ecken hat und $\chi(G) \leq 2$ falls $G$ bipartit ist. 
Ein Graph ist \DF{bipartit}, falls es zwei disjunkte Mengen $A,B\subset V(G)$ existieren, sodass $V(G) = A\cup B$ und $G[A], G[B]$ besitzen keine Kanten gilt.
Nach dem Satz von K"onig ist $G$ genau dann bipartit, wenn $G$ keinen ungeraden Kreis als Untergraphen besitzt (\cite{koenig}). 

Nach Stockmeyer \cite{stockmeyer} ist f"ur jedes $k\geq 3$ das Entscheidungsproblem ob ein gegebener Graph $k$ f"arbbar ist {\sf NP}-vollst"andig. Es ist also nicht zu erwarten, dass sich Graphen mit chromatischer Zahl kleiner gleich $k$ f"ur festes $k\geq 3$ einfach charakterisieren lassen.

Der folgende Satz stammt von Brooks aus dem Jahr $1941$. Die Schranke werden wir sp"ater noch verbessern, siehe Satz \ref{thm:wilfev}. 
\begin{theorem}[Brooks]
  Sei $G$ ein zusammenh"angender Graph mit Maximalgrad $\Delta$. Dann gilt $$\chi(G) \leq \Delta + 1. $$
  Gleichheit tritt nur auf, falls $G$ ein vollst"andiger Graph oder ein ungerader Kreis ist.
  \label{thm:brooks}
\end{theorem}
Bei der Untersuchung des F"arbungsproblems f"ur Graphen erweisen sich die kritischen Graphen als ein n"utzliches Hilfsmittel. Dies liegt vor allem daran, dass sich F"arbungsprobleme f"ur Graphen oft auf entsprechende F"arbungsprobleme f"ur kritische Graphen zur"uckf"uhren lassen.
Ein Graph $G$ hei{\ss}t \DF{k-kritisch}, falls $\chi(G) = k$ ist und $\chi(H) < k$ gilt f"ur alle echten induzierten Untergraphen $H$ von $G$. 

Ist $G$ ein Graph und $t\in V(G) \cup E(G)$, so gilt 
$$ \chi(G) -1 \leq \chi(G-t) \leq \chi(G).$$
Daraus erhalten wir den folgenden bekannten Satz, wonach die $(k-1)$-f"arbbaren Graphen durch verbotene $k$-kritische Untergraphen charakterisiert werden k"onnen.

\begin{theorem}
  Sei $G$ ein Graph und $k\in\N$. Dann ist $\chi(G) \geq k$, genau dann wenn $G$ einen $k$-kritischen Graphen $H$ als induzierten Untergraphen enth"alt.
  \label{thm:forbiddencritgraph}
\end{theorem}

\begin{proof}
  Falls $G$ einen $k$-kritischen Untergraphen enth"alt, so ist $\chi(G) \geq \chi(H) = k$, da $\chi$ ein monotoner Graphenparamter ist.

  Sei also $\chi(G) \geq k$. Wir w"ahlen $H$ als einen induzierten Untergraphen von $G$ kleinster Ordnung, sodass $\chi(H) \geq k $ gilt. Dann ist $\chi(H-v) < k$ f"ur alle Ecken von $H$. Also ist $H$ $k$-kritisch. 
\end{proof}
\subsection{F"arbungen von Kantengraphen}
In diesem Abschnitt betrachten wir das F"arbungsproblem f"ur die Klasse der Kantengraphen. 
Der \DF{Kantengraph} $L(H)$ eines Hyergraphen $H$ ist der Graph mit der Eckenmenge $V(L(H))= E(H) $ und Kantenmenge $$E(L(H)) = \left\{ ee'| \left\{ e,e' \right\} \in [E(H)]^{2}, e\cap e' \neq \emptyset \right\}.$$
Zwei verschieden Kanten von $H$, welche eine Ecke gemeinsam hei�en adjazent. 
F"ur eine Kante $e$ von $H$ sei $d_{H}(e) = d_{L(H)}(e) $ der \DF{Kantengrad} von $e$ in $H$. Dieser ist also die Zahl der von $e$ verschiedenen Kanten $e'$ von $H$, welche mit $e$ adjazent sind. 
Dann sei $\Delta'(H)$ der \DF{maximale Kantengrad} von $H$ und $\delta'(H)$ der \DF{minimale Kantengrad} und wir setzen $\Delta'(H) = \delta'(H)=0$, falls $E(H) = \emptyset$. 

Eine F"arbung des Kantengraphen eines Hypergraphen $H$, hei�t \DF{Kantenf"arbung} von $H$. Wir bezeichnen dann mit dem \DF{chromatischen Index} $\chi'(H)$ die kleinste nat"urliche Zahl $k$, sodass $L(H)$ eine $k$-F"arbung besitzt. Also gilt $\chi'(H) = \chi(L(H))$. 
Ist $v$ eine Ecke von $H$, so ist $E_H(v)$ eine Clique von $L(H)$, und folglich gilt $\chi'(H) \geq \Delta(H)$. K"onig zeigte in \cite{koenigzwei}, dass diese Schranke f"ur bipartite Graphen scharf ist.
\begin{theorem}[K"onig]
  Ist $G$ ein bipartiter Graph, so gilt $\chi'(G) = \Delta(G)$.
  \label{}
\end{theorem}

\begin{theorem}[Vizing]
  Sei $G$ ein Graph mit Maximalgrad $\Delta$. Dann gilt $\chi'(G) = \Delta$ oder $\chi'(G) = \Delta + 1$. 
  \label{thm:Vizing}
\end{theorem}

\subsection{Eigenwerte von symmetrischen Matrizen}
\label{sss:ewgraph}
\label{ssec:EigenwerteMatrizen} 
Bevor wir uns den Eigenwerten von Graphen zuwenden, wollen wir den Leser an einige bekannte Tatsachen "uber symmetrische Matrizen erinnern. Es sei $A\in \Rnn$ eine symmetrische Matrix der Ordnung $n\in \N$. Dann ist 
\begin{equation}
  Ax = \lambda x
  \label{eq:eigenwertgleichung}
\end{equation}
mit $x\in \R^{n}$ und $\lambda\in\R$ die (reelle) Eigenwertgleichung von $A$. F"ur $\lambda\in\R$ ist die L"osungsmenge 
\begin{equation*}
  E_{A}(\lambda) = \left\{ x\in\R^{n} | Ax = \lambda x\right\}
\end{equation*}
ein linearer Unterraum von $\R^{n}$ mit $\operatorname{dim}(E_A(\lambda) \geq 0$. Man nennt dann $\lambda$ einen \DF{Eigenwert} von $A$, falls $\operatorname{dim}(E_A(\lambda)) \geq 1$ ist, die Vektoren aus $E_A\left( \lambda \right)$ hei{\ss}en \DF{Eigenvektoren} von $A$ zum Eigenwert $\lambda$ und $E_A\left( \lambda \right)$ ist der zu $\lambda$ geh"orende \DF{Eigenraum} von $A$. Die Abbildung $p_A$ mit $$p_A(\lambda) = \det (A-\lambda I)$$ ist ein Polynom aus $\R[\lambda]$ vom Grade
$n$, welches \DF{charakteristisches Polynom} von $A$ genannt wird (dabei ist $I\in\Rnn$ die Einheitsmatrix der Ordnung $n$). 
F"ur $\lambda\in\R$ gilt dann $$\lambda \text{ ist Eigenwert von }A \Leftrightarrow p_A(\lambda) = 0.$$ Da $A$ symmetrisch ist, zerf"allt $p_A$ in genau $n$ reelle Linearfaktoren, d.h. $p_A$ hat genau $n$ Nullstellen (gez"ahlt mit ihren Vielfachheiten). 
F"ur $\lambda\in\R$ sei $m_A(\lambda$ die Vielfachheit von $\lambda$ als Nullstelle von $p_A$. Die Matrix $A$ besitzt somit $n$ reelle Eigenwerte, welche wir monton fallend anordnen k"onnen. Im Folgenden bezeichnen wir mit $\lambda_{p}(A)$ den $p$-gr"o�ten Eigenwert von $A$, das hei�t es gilt $$\lambda_{1}(A) \geq \lambda_2(A) \geq \dots \lambda_{n}(A)$$ Dann ist $\lambda_{max}(A) = \lambda_1(A)$ der gr"o�te Eigenwert von $A$ und $\lambda_{min}(A) = \lambda_{n}(A)$ der
kleinste Eigenwert von $A$. Die Folge $$\operatorname{sp}(A) = (\lambda_{1}(A),\lambda_{2}(A), \dots \lambda_{n}(A))$$ der Eigenwerte bezeichnet man als das \DF{Spektrum} von $A$. Bekanntlich besitzen zwei symmetrische Matrizen genau dann das gleiche Spektrum, wenn sie zueinander "ahnlich sind.
Ist $\lambda$ ein Eigenwert von $A$ so ist $\operatorname{dim}E_A(\lambda) = m_A(\lambda)$. Eigenvektoren zu verschiedenen Eigenwerten sind stets orthogonal und $\R^{n}$ ist die direkte Summe der Eigenr"aume zu den verschiedenen Eigenwerten. Insbesondere besitzt der $\R^{n}$ eine Orthonomalbassi aus lauter Eigenwerten von $A$.

Eine symmetrische Matrix $A\in\Rnn$ hei�t positiv semidefinit, falls $x^{T}Ax \geq 0$ f"ur alle Vektoren $x\in \R^{n}$ gilt. Gilt zus"atzlich noch $x^{T}Ax = 0$ nur f"ur $x= 0\in\R^{n}$ (hat also $A$ vollen Rang), so hei�t $ A$ positiv definit. Wir wollen nun einige Eigenschaften von positiv (semi)definiten Matrizen anf"uhren.
\begin{proposition}
  Folgende Aussagen sind f"ur eine symmetrische Matrix $A\in \Rnn$ "aquivalent
  \begin{enumerate}[label={\rm(\alph*)}]
    \item $A$ ist positiv semidefinit.
    \item Alle Eigenwerte von $A$ sind nicht negativ.
    \item $A=UU^T$ f�r eine Matrix $U\in\R^{n\times m}$.
  \end{enumerate}
  \label{prop:psdmatrix}
\end{proposition}

F"ur $A\in\Rnn$ und $x\in \R$ sei der \DF{Rayleigh-Quotient} $R_A(x)$ definiert als $$R_A(x) = \frac{x^TAx}{x^Tx}.$$
Der folgende Satz findet sich in der Literatur als Courant-Fischer Minmax Theorem, es scheint schwierig zu sein eine Originalquelle zu finden \todo{finden}. 
\begin{theorem}[Courant-Fischer]
  Sei $A\in \R^{n\times n}$ symmetrisch. Dann gilt f"ur alle $p\in\{1,\dots, n\}$:
  \begin{enumerate}[label={\rm(\alph*)}]
    \item $\lambda_p(A) = \max\{\min\limits_{x\in V , x \neq 0} R_A(x)| V\subseteq \R^n \text{ ist linearer Unterraum der Dimension } p\}$.
    \item $\lambda_p(A) = \min\{\max\limits_{x\in V , x \neq 0} R_A(x)| V\subseteq \R^n \text{ ist linearer Unterraum der Dimension } n-p+1\}$.
  \end{enumerate}
  \label{thm:CourantFischer}
\end{theorem}

\begin{lemma}
  Seien $A,B\in \Rnn$ symmetrisch und $A-B$ positiv semidefinit. Dann ist $\lambda_p(A)\geq\lambda_p(B)$ f"ur alle $1\leq p \leq n$.
  \label{lem:evpsddif}
\end{lemma}

\begin{proof}
  Sei $x\in\R^{n}\setminus \left\{ 0 \right\}$ beliebig. Dann gilt $x^{T}(A-B)x \geq 0$, da $A-B$ positiv semidefinit ist. Daraus folgt
  \begin{align*}
    x^{T} A x &\geq x^{T} B x
  \end{align*}
  und folglich ist $\frac{x^{T} A x}{x^{T}x} \geq\frac{x^{T} B x}{x^{T}x}$ und es folgt mit Satz \ref{thm:CourantFischer}(a) : 
  \begin{align*}
    \lambda_p(A) &= \max\{\min\limits_{x\in V , x \neq 0} \frac{x^TAx}{x^Tx}| V\subseteq \R^n \text{ ist linearer Unterraum der Dimension } p\}\\
    &\geq \max\{\min\limits_{x\in V , x \neq 0} \frac{x^TBx}{x^Tx}| V\subseteq \R^n \text{ ist linearer Unterraum der Dimension } p\} \\
    &= \lambda_p(B)
  \end{align*}
  f"ur $1\leq p \leq n$. 
\end{proof}
\begin{theorem}[Interlacing]
  \label{thm:Interlacing}
  Sei $A\in \R^{n\times n}$ symmetrisch und sei $B\in\R^{(n-k)\times(n-k)}$ die symmetrische Matrix, welche aus $A$ durch L"oschen von Zeilen und den entsprechenden Spalten entsteht. Dann ist $B$ symmetrisch, und es gilt:
  \begin{align*}
    \lambda_{p}(A) \geq \lambda_{p}(B) \geq \lambda_{p+k}(A) 
  \end{align*}
  f"ur $p = 1,\dots n-k$.
\end{theorem}

\begin{proof}
  Seien $l_1<\dots< l_{n-k}$ die Nummern der Zeilen bzw. Spalten die nicht gel"oscht werden. Setze $P=(e_{l_1},e_{l_2},\dots,e_{l_{n-k}})\in \R^{n\times \left( n-k \right)}$, wobei $e_{k}$ der $k$-te Einheitsvektor des $\R^{n}$ ist. Dann besitzt $P$ vollen Spaltenrang und es gilt $B=P^{T}AP$. 
  Seien $V \subseteq\R^{\left( n-k \right)}$ ein linearer Unterraum ,  $x\in  V $ beliebig und $y = Px$. Dann ist $y\in \operatorname{im} P|_{V} = \{z\in \R^{n}| z = Px, x \in V \}$ und es gilt $y^{T}y = x^{T}P^{T}Px = x^{T}x$, da $P^{T}P = I_{n_k}$ ist.
  Au�erdem ist $\operatorname{im} P|_{V}$ ein linearer Unterraum des $\Rn$ mit $\operatorname{dim} (\operatorname{im} P|_{V}) = \operatorname{dim} (V)$, da $P$ vollen Spaltenrang besitzt. Mit Satz \ref{thm:CourantFischer}(a) folgt f"ur $1 \leq p  \leq n-k$ :
  \begin{align*}
    \lambda_{p}(B) &= \max\{\min\limits_{x\in V , x \neq 0} \frac{x^TBx}{x^Tx}| V\subseteq \R^{\left( n-k \right)} \text{ ist linearer Unterraum der Dimension } p\} \\
    &= \max\{\min\limits_{x\in V , x \neq 0} \frac{x^TP^{T}APx}{x^Tx}| V\subseteq \R^{\left( n-k \right)} \text{ ist linearer Unterraum der Dimension } p\} \\
    &= \max\{\min\limits_{y\in \operatorname{im} P|_{V} , y \neq 0} \frac{y^{T}Ay}{y^Ty}| V\subseteq \R^{\left( n-k \right)} \text{ ist linearer Unterraum der Dimension } p\} \\
    &\leq \max\{\min\limits_{x\in W , x \neq 0} \frac{y^TAy}{y^Ty}| W\subseteq \R^n \text{ ist linearer Unterraum der Dimension } p\} \\
    &= \lambda_p(A)
  \end{align*}
  Damit ist die erste Ungleichung gezeigt. Die zweite folgt analog bei Betrachtung von $-A$ und $-B$, da $\lambda_{p}(-A) = -\lambda_{n-p+1}(A)$ ist.
\end{proof}

Die folgenden Ungleichungen werden sp"ater bei der Betrachtung der Eigenwerte von Graphen hilfreich seien. Ein Beweis f"ur die Weyl Ungleichungen findet sich in \cite[6.7]{zbMATH03278338}.
\begin{theorem}[Weyl Ungleichungen]
  Seien $A,B,C\in\Rnn$ symmetrische Matrizen mit $A = B+C$. Dann gilt f"ur alle $1\leq p \leq n$
  \begin{equation*}
    \lambda_{p}(B) + \lambda_{n}(C) \leq \lambda_{p}(A) \leq \lambda_{p}(B) + \lambda_{1}(C)
  \end{equation*}
  \label{thm:weylineq}
\end{theorem}
Ein Beweis f"ur die folgenden Ungleichungen findet sich in \cite[3.]{1108.1467}.
\begin{theorem}[Ky Fan Ungleichungen]
  Seien $A,B,C\in\Rnn$ symmetrische Matrizen mit $A = B+C$. Dann gilt f"ur alle $k\leq n$
  \begin{equation*}
    \sum\limits_{p=1}^{k} \lambda_{p}(A) \leq \sum\limits_{p=1}^{k} \lambda_{p}(B) + \sum\limits_{p=1}^{k} \lambda_{p} (C)
  \end{equation*}
  F"ur $k=n$ gilt Gleichheit.
  \label{thm:kyfanineq}
\end{theorem}

\subsection{Eigenwerte von Graphen}

\label{ssec:EigenwerteGraphen}
Sei $G$ ein Graph der Ordnung $n$ mit der Eckenmenge $V(G) = \left\{ v_1,v_2,\dots,v_n \right\}$. Die \DF{Adjanzenzmatrix} von $G$ ist die Matrix $ A(G)\in \Rnn$ mit \[A(G)_{ij} = \begin{cases}
    1 & \text{falls }v_{i}v_{j} \in E(G) \\ 0 & \text{falls }v_{i}v_{j} \notin E(G)
\end{cases}\] 
Dann ist $A$ symmetrisch, und hat folglich nur reelle Eigenwerte.
Damit es Sinn ergibt, von den Eigenwerten eines Graphen zu sprechen, d"urfen die Eigenwerte von $A(G)$ nicht von der Nummerierung der Ecken abh"angen. Das dem so ist, zeigt das folgende Lemma.

\begin{lemma}
  Sei $G=(V,E)$ ein Graph. Dann ist das charakteristische Polynom von $A(G)$ unabh"angig von der Nummerierung der Ecken von $G$. 
  \label{lem:GraphEigenwerte}
\end{lemma}

\begin{proof}
  Seien $V(G)=\{v_1,\dots, v_n\}= \{u_1,\dots, u_n\}$ zwei Nummerierungen der Ecken. Sei weiterhin $A,B\in \R^{n\times n}$ mit  
  \[
    A_{ij} = \begin{cases}
      1 & \text{falls }v_iv_j \in E(G) \\ 0 & \text{falls }v_iv_j \notin E(G)
    \end{cases} \\ \mbox{ und }
    B_{ij} = \begin{cases}
      1 & \text{falls }u_iu_j \in E(G) \\ 0 & \text{falls }u_iu_j \notin E(G)
  \end{cases}\] 
  Dann gibt es eine Permutation $\sigma\in S^{n}$ sodass $v_{\sigma(i)} = u_i$ ist. Folglich gilt $A_{\sigma(i),\sigma(j)} = B_{ij}$ . Sei $P \in GL_n(\R) $ die Matrix mit
  \[
    P_{ij} = \begin{cases}
      1 & \text{falls }\sigma(i) = j \\ 0 & \text{sonst} 
    \end{cases} 
  \]
  Damit ist $P=(e_{\sigma(1)},\dots, e_{\sigma(n)})$. F"ur die Matrix $P^{T}AP$ gilt dann: 
  \[
    (P^{T}AP)_{ij} = e_j^{T} P^{T}AP e_i = e_{\sigma(j)}^{T}Ae_{\sigma(i)}=A_{\sigma(i),\sigma(j)} = B_{ij}
  \]
  Also ist $P^{T}AP = B$. Somit sind $A$ und $B$ "ahnlich, und besitzen folglich das selbe charakteristische Polynom. 
\end{proof}
F"ur den Graphen $G$ der Ordnung $n$ sei dann $p_G = p_{A(G)}$ das \DF{charakteristische Polynom} von $G$ und $\lambda_p(G) = \lambda_{p}(A(G))$ der \DF{$p$-gr"o�te Eigenwerte} von $G$ (f"ur $1\leq p \leq |G|$) und $\operatorname{sp}(G) = \operatorname{sp}(A(G))$ das \DF{Spektrum} von $G$. Diese sind nach Lemma \ref{lem:GraphEigenwerte} unabh"angig von der Nummerierung der Ecken, und somit wohldefiniert. Dies gilt jedeoch nicht f"ur die Eigenvektoren. 
Wir k"onnen aber auch eine koordinatenfreie Interpretation f"ur die Eigenvektoren geben. Dazu betrachten wir einen Eigenwert $\lambda$ von $G$ und einen zugeh"origen Eigenvektor $x$ von $A(G)$. Aus der Eigenwertgleichung \ref{eq:eigenwertgleichung} erhalten wir das Gleichungssystem 
\begin{equation}
 \lambda x_{i} = \sum\limits_{j=1}^{n}A(G)_{ij}x_j 
  \label{eq:ewglneu}
\end{equation}
f"ur $1\leq i \leq n$. Der Vektor $x\in \R^{n}$ ordnet der Ecke $v_i$ den Wert $x_i = x(v_i)$ zu und die Gleichung \ref{eq:ewglneu} ist "aquivalent zu 
\begin{equation}
  \lambda x(v_i) = \sum\limits_{v_j: v_iv_j\in E(G)} x(v_j).
  \label{eq:eqglneu2}
\end{equation}
Wir betrachten nun den Vektorraum $\R^{V(G)}$ aller Abbildungen $x:V(G) \to \R$. Offenbar ist die Abbildung $x\in \R^{V(G)}$ genau dann ein Eigenvektor von $G$ zum Eigenwert $\lambda$, wenn f"ur alle Ecken $v$ von $G$ gilt:
\begin{equation}
  \lambda x(v) = \sum\limits_{u: uv\in E(G)} x(u)
  \label{eq:ewglneu3}
\end{equation}
d.h. die Summe der Werte $x(u)$ "uber die Nachbarn $u$ von $v$ ergibt den Wert $\lambda x(v)$. Es sei dann $E_G(\lambda)$ die Menge aller dieser Abbildungen. Dann ist $E_G(\lambda)$ der \DF{Eigenraum} von $G$ zum Eigenwert $\lambda$. Es sei $\mathbf{ 1 } \in \R^{V(G)}$ die Abbildung mit $\mathbbm 1 (v) = 1$ f"ur alle $v\in V(G)$. F"ur eine Ecke $v$ von $G$ gilt dann 
\begin{equation*}
  d_G(v) = \sum\limits_{u:uv\in E(G)} 1 = \sum\limits_{u:uv\in E(G)} \mathbbm 1 (u)
\end{equation*}
und aus Gleichung \ref{eq:ewglneu} folgt somit f"ur $r\in \N$ \todo{$N_0$?}:
\begin{equation}
  \mathbbm 1 \in E_G(r) \Leftrightarrow G \text{ ist } r \text{-regul"ar}.
  \label{eq:Grregul}
\end{equation}

Zwei Graphen hei{\ss}en \DF{isospektral}, falls sie das selbe Spektrum besitzen.

\begin{remark}
  Zwei isospektrale Graphen sind nicht unbedingt isomorph.
\end{remark}

\begin{proof}
 Betrachte dazu folgende Graphen :
 \todo{bild}

 Diese sind offenbar nicht isomorph. Betrachten wir die Adjazenzmatrizen der beiden Graphen 
 \begin{align*}
   A(G) = \begin{pmatrix}
     0 & 1 & 1 & 1 & 1 \\
     1 & 0 & 0 & 0 & 0 \\
     1 & 0 & 0 & 0 & 0 \\
     1 & 0 & 0 & 0 & 0 \\
     1 & 0 & 0 & 0 & 0 
   \end{pmatrix}  \text{ und }A(H) = \begin{pmatrix}
     0 & 0 & 0 & 0 & 0 \\
     0 & 0 & 1 & 0 & 1 \\
     0 & 1 & 0 & 1 & 0 \\
     0 & 0 & 1 & 0 & 1 \\
     0 & 1 & 0 & 1 & 0 
   \end{pmatrix}
 \end{align*}
 und berechnen die Spektra, so erkennen wir das beide Graphen das Spektrum $$\operatorname{sp}(G)=\operatorname{sp}(H) = (2,0,0,0,2)$$ besitzen. Also sind $G$ und $H$ isospektral, aber nicht isomorph.
\end{proof}

\begin{example}
  \label{ex:vollstgraph}
  F"ur den vollst"andigen Graphen der Ordnung $K_n$ der Ordnung $n\geq 1$ gelten folgenden Aussagen:
  \begin{enumerate}[label=\rm{(\alph*)}]
    \item $\operatorname{sp}(K_n) = (n-1, \underbrace{-1,-1,\dots,-1)}_{n-1 \text{ mal}}$.
    \item $p_{K_n}(\lambda) = (-1)^{n}(\lambda-(n-1))(\lambda+1)^{n-1}$.
    \item $ E_{K_n}(n-1) = [\mathbbm 1]$, wobei $\mathbbm 1\in \R^{V(K_n)}$ die Einsabbildung ist mit $1(v) = 1$ f"ur alle $v\in V(G)$.
    \item $ E_{K_n}(-1) = \left\{ x\in \R^{V(K_n)}| x \text{ ist orthogonal zu } \mathbbm 1 \right\}$.
  \end{enumerate}
\end{example}

\begin{proof}
  Es sei $J\in\Rnn$ die Matrix, welche nur $1$ als Eintr"ag besitzt (diese werden wir mit $1$-Matrix bezeichnen). Dann gilt $A(K_n) = J-I$, wobei $I\in\Rnn$ die Einheitsmatrix ist. Da $\operatorname{rang}(J) = 1$, ist also $-1$ ein Eigenwert von $G$ mit Vielfachheit $n-1$. Da $K_n$ $n-1$ regul"ar ist, ist auch $n-1$ ein Eigenwert von $K_n$. Damit folgen (a) und (b). Weiterhin gilt (c) wegen Gleichung \ref{eq:Grregul}. Aus der Orthognalit"at der Eigenvektoren folgt nun (d).
\end{proof}

\begin{example}
  Der Kreis $C_n$ der Ordnung $n\geq 3$ hat die Eigenwerte $$\lambda_p (C_n) = 2 \cos\left(\frac{2\pi p}{n}\right)$$ f"ur $1\leq p \leq n$. Einen Beweis findet der Leser in \cite[1.1.4]{zbMATH05625877}. Insbesondere gilt 
  \begin{align*}
    \operatorname{sp}(C_3) = \operatorname{sp}(K_3) &= (2,-1,-1) \\
    \operatorname{sp}(C_4) &= (2,0,0,-2) \\
    \operatorname{sp}(C_5) &= (2, \frac{1}{2}(\sqrt 5 - 1),\frac{1}{2}(\sqrt 5 - 1),\frac{1}{2}(-\sqrt 5 - 1),\frac{1}{2}(-\sqrt 5 - 1))
  \end{align*}
\end{example}

\begin{example}
  F"ur den kantenlosen Graphen $O_n$ der Ordnung $n\in\N$ ist $A(O_n)$ die Nullmatrix und somit gilt $p_{O_n}(\lambda) = (-1)^{n} \lambda^{n}$ und $\lambda_{i} (O_n) = 0$ f"ur alle $1\leq i \leq n$.
\end{example}

\begin{example}
  Ist $G$ die disjunkte Vereinigug der nichtleeren Graphen $G_1,G_2,\dots, G_l$ mit $l\in \N$, so ist $$p_G(\lambda) = p_{G_1}(\lambda) \cdot p_{G_2}(\lambda) \cdot \dots \cdot p_{G_l}(\lambda)$$ f"ur $\lambda\in\R$. Das Spektrum von $G$ ergibt sich somit aus der Vereinigung der Spektren von $G_1,G_2,\dots,G_l$ und f"ur $\lambda \in \R$ gilt:
  $$m_G(\lambda) = m_{G_1}(\lambda)+ m_{G_2}(\lambda)+\dots+ m_{G_l}(\lambda) $$
\end{example}

\begin{proof}
  Wir nummerieren die Ecken von $G$ so, dass f"ur $1\leq i \leq l-1$ die Ecken von $G_i$ vor den Ecken von $G_{i+1}$ aufgelistet werden. Dann ist $$A(G) = \operatorname{diag}(A(G_1),A(G_2),\dots,A(G_l))$$ und somit ist $$A(G) - \lambda I = \operatorname{diag}(A(G_1)-\lambda I,A(G_2)-\lambda I,\dots,A(G_l)-\lambda I)$$ wobei $I$ die Einheitsmatrix der passenden Ordnung ist. Dann ist 
  \begin{align*}
    p_{G}(\lambda) &= \det(A(G) - \lambda I) \\
    &= \det(A(G_1)-\lambda I) \cdot\det(A(G_2)-\lambda I)\cdot \ldots \cdot\det(A(G_l)-\lambda I) \\
    &= p_{G_1}(\lambda) \cdot p_{G_2}(\lambda) \cdot \ldots \cdot p_{G_l}(\lambda) \cdot
  \end{align*}
  Da die Eigenwerte die Nullstellen des charakteristischen Polynoms sind, gilt dann f"ur alle $\lambda \in \R$: 
  $$m_G(\lambda) = m_{G_1}(\lambda)+ m_{G_2}(\lambda)+\dots+ m_{G_l}(\lambda) .$$
\end{proof}

Ist $G$ die disjunkte Vereinigung zweier vollst"andiger Graphen $K_n$, so ist $$\operatorname{sp}(G) = (n-1,n-1, \underbrace{-1,\dots,-1}_{2n-2 \text{ mal}}).$$ Insbesondere ist $\lambda_{max}(G) = n-1$ ein doppelter Eigenwert von $G$.

\subsection{Eigenschaften des Graphenspektrums}
In diesem Abschnitt wollen wir einige einfache, aber wichtigen Eigenschaften der Spektra von Graphen geben.
Eine Matrix $A\in\Rnn$ hei�t \DF{reduzibel}, falls zwei nichtleere disjunkte Teilmengen $X,Y$ von $\left\{ 1,\dots,n \right\}$ existieren, mit $X\cup Y = \left\{ 1,\dots,n \right\}$ und $$A_{ik} = 0 ~ i\in X, k \in Y.$$ Anderfalls hei�t $A$ \DF{irreduzibel}. 
Wie man leicht zeigen kann, ist $A$ reduzibel, genau dann, wenn man durch Umordnen der Zeilen und Spalten in folgende Blockform gebracht werden kann: $$\begin{pmatrix}
  A_{11} & 0 \\ A_{12} & A_{22}
\end{pmatrix}.$$ 
Ist $G$ ein Graph, so gilt $$G \text{ ist zusammenh"angend} \Leftrightarrow A(G) \text{ist irreduzibel}.$$
Im Jahr 1912 bewiesen Perron und Frobenius (\cite{perron}) einen zentralen Satz "uber die Eigenwerte von unzerlegbaren Matrizen mit nicht negativen Elementen. Der folgende Satz ist eine unmittelbare Folgerung aus dem Satz von Perron-Frobenius.

\begin{theorem}
  F"ur einen zusammenh"angenden Graphen $G$ der Ordnung $n\in \N$ gelten folgende Aussagen:
  \begin{enumerate}[label=\rm{(\alph*)}]
    \item $\lambda_{max}(G)$ ist ein einfacher Eigenwert mit $\lambda_{max}(G) \leq \Delta(G)$.
    \item Es gibt einen Eigenvektor $x$ von $G$ zum Eigenwert $\lambda_{max}(G)$ mit $x(v) > 0$ f"ur alle $v\in V(G)$. 
    \item Ist $x$ ein Eigenvektor von $G$ zum Eigenwert $\lambda$ mit $x(v)> 0$ f"ur alle $v\in V(G)$, so ist $\lambda = \lambda_{max}(G)$.
    \item F"ur alle Eigenwerte $\lambda$ von $G$ gilt $|\lambda| \leq \lambda_{max}(G)$.
  \end{enumerate}
\end{theorem}

\begin{corollary}
  F"ur einen $r$-regul"aren Graphen $G$ der Ordnung $n \in\N$ gelten folgende Aussagen:
  \begin{enumerate}[label=\rm{(\alph*)}]
    \item $\lambda_{max}(G) = r$.
    \item $G$ ist genau dann zusammenh"angend, wenn $\lambda_{max}(G)$ ein einfacher Eigenwert ist.
  \end{enumerate}
\end{corollary}

\begin{proof}
  Es seien $G_1,\dots,G_l$ die Komponenten von $G$. Dann ist jede Komponente $G_i$ ein $r$-regul"arer Graph ($1\leq i \leq l$). Damit folgt aus \ref{eq:Grregul}, dass $\mathbbm 1$ ein Eigenvektor von $G_i$ zum Eigenwert $r$ ist. Dann ist $\lambda_{max}(G) = r$ und $m_{G}(r) = l$. Somit gilt sowohl (a) als auch (b).
\end{proof}

F"ur einen Graphen $G$ sei $\overline{G}$ der \DF{Komplementargraph} von $G$ mit $V(\overline{G}) = V(G)$ und $E(\overline{G}) = [V(G)]^{2}\setminus E(G)$.

\begin{corollary}
  Ist $G$ ein $r$-regul"arer Graph der Ordnung $n\geq 1$, so gelten folgende Aussagen:
  \begin{enumerate}[label=\rm{(\alph*)}]
    \item $\lambda_{1}(G) = r $ und $\lambda_{1}(\overline{G}) = n-1-r $.
    \item $\lambda_{i}(\overline{G}) = - \lambda_{n-i+1}(G) -1$ f"ur $2\leq i \leq n$. 
  \end{enumerate}
\end{corollary}

\begin{proof}
  Aussage (a) folgt, da der Komplementargraph eines $r$-regul"aren Graphes $n-1-r$-regul"ar ist. 
  Zum Beweis von (b) w"ahlen wir f"ur $G$ und $\overline{G}$ die selbe Nummerierung der Ecken, etwa $v_1,v_2,\dots v_n$. Dann ist $$A(G) + A(\overline{G}) = J-I$$ wobei $J \in \Rnn$ die $1$-Matrix ist, und $I\in\Rnn$ die Einheitsmatrix ist. Wir betrachten $\lambda = \lambda_i(G)$ f"ur $i \geq 2$. Dann existiert ein Eigenvektor $x$ zum Eigenwert $\lambda$. Dieser ist orthogonal zu $\mathbbm 1 \in E_{G}(r)$. Folglich ist 
  \begin{align*}
    A(G)x + A(\overline{G})x = Jx -Ix = -x
  \end{align*}
  Durch Umstellen erhalten wir $A(\overline{G})x = (-\lambda-1)x$. Also ist $(-\lambda-1)$ ein Eigenwert von $\overline{G}$. Analog k"onnen wir zeigen, dass f"ur $\lambda = \lambda_{i}(\overline{G})$  ($i\geq 2$) $-1-\lambda$ ein Eigenwert von $G$ ist. 
\end{proof}

\begin{lemma}
  \label{rem:evGraph} 

  Sei $G$ ein Graph der Ordnung $n\in\N$ mit $m=|E(G)|$ Kanten.
  \begin{enumerate}[label={\rm(\alph*)}]
    \item Die Summe aller Eigenwerte von $G$ (mit Vielfachheiten) ist $0$. 
    \item Die Summe der Quadrate aller Eigenwerte von $G$ (mit Vielfachheiten) ist $2m$.
  \end{enumerate}
\end{lemma}

\begin{proof}
  Wir zeigen zun"achst (i). F"ur die Adjazenzmatrix $A = A(G)$ gilt $\operatorname{spur}(A)  = \sum\limits_{i=1}^{n} a_{ii}= 0$. Aus der Linearen Algebra ist bekannt, dass $\sum\limits_{i=1}^{n}\lambda_{i}(A) = \operatorname{spur}(A) = 0$ ist.

  Um (ii) zu beweisen, betrachten wir $B=A^{2}$. Die Eintr"age $B_{ij}$ geben die Anzahl aller Kantenfolgen der L"ange $2$ zwischen den Ecken $v_i$ und $v_j$ an. Insbesondere gilt $B_{ii} = d_{G}(v_i)$, da jede Kantenfolge der L"ange $2$ von $v_i$ nach $v_i$ genau einer Kante entspricht. Daraus folgt:

  \begin{equation*}
    \sum\limits_{i=1}^{n} \lambda_{i}(G) ^{2} = \operatorname{spur}(B) = \sum\limits_{i=1}^{n} B_{ii} \sum\limits_{i=1}^{n} d_{G}(v_{i}) = 2|E(G)|
  \end{equation*}
\end{proof}

\begin{lemma}
  Seien $H$ ein induzierter Untergraph von $G$ und $k = |G| - |H|$. Dann gilt
  $$ \lambda_p(G) \geq \lambda_{p}(H) \geq \lambda_{p+k}(G)$$
  f"ur $1\leq p \leq n-k$.
  \label{lem:InterlacingGraphen}
\end{lemma}

\begin{proof}
  Ist $H$ ein induzierter Untergraph von $G$, so entsteht $A(H)$ aus $A(G)$ durch Streichen von Spalten und den korrespondierenden Zeilen. Damit folgt die Behauptung aus Satz \ref{thm:Interlacing}.
\end{proof}

\begin{corollary}
  Sei $G$ ein Graph mit $\omega(G) = p$ und $\alpha(G) = q$. Dann gilt:
  \begin{equation*}
    \lambda_p(G) \geq -1 \mbox{ und } \lambda_q(G) \geq 0 \text{.}
  \end{equation*} \label{cor:alphaomegaEigenwerte}
\end{corollary}
\begin{proof}
  Ist $\omega(G) = p$, so besitzt $G$ einen vollst"andigen induzierten Untergraphen $H$, der Ordnung $p$. Dann gilt $\lambda_{1}(H) = p-1$ und $\lambda_{i}(H) = -1$ f"ur $2\leq i \leq p$(siehe Beispiel \ref{ex:vollstgraph}). Damit folgt aus Korollar \ref{cor:alphaomegaEigenwerte}, dass $\lambda_{p}(G) \geq \lambda_{p}(H) \geq -1$ ist. 
  
  Ist $\alpha(G) = q$, so besitzt $G$ einen kantenlosen induzierten Untergraphen $H'$ der Ordnung $q$.
  Dann ist $\lambda_{i} (H') = 0$ f"ur  $1\leq i \leq q$. Also folgt aus Korollar \ref{cor:alphaomegaEigenwerte}, dass $\lambda_{p}(G) \geq \lambda_{p}(H') = 0$ ist.
\end{proof}
