\section{Einf"uhrung}
\label{sec:Einf"uhrung}
Gegenstand dieser Bachelorarbeit ist \todo{gute Einleitung, Verweis auf EFL Resultate}
\subsection{Graphentheoretische Grundlagen}
\label{ssec:graphengrundlagen}
Es sei $G=(V,E)$ stets ein (schlichter, ungerichteter) Graph mit Eckenmenge $V=V(G) = \left\{ 1,\dots,n \right\}$ und Kantenmenge $E=E(G)= \left\{ ij| i \text{ und } j \text{ sind in } G \text{ adjazent} \right\}$. Der \DF{Grad} einer Ecke $v\in V(G)$ sei definiert als $d_{G}(v) = \left\{ j|ij\in E(G) \right\}$. Der \DF{Minimalgrad} (\DF{Maximalgrad}) sei definiert als der kleinste (gr"o{\ss}te) Grad einer Ecke von $G$ und wird mit $\delta(G)$ ($\Delta(G)$) bezeichnet.

Wir wollen nun einige h"aufig auftretenden Graphen bezeichnen. Der \DF{vollst"andige Graph} der Ordnung $n$ sei $K_n$ und sein Komplement, der \DF{leere Graph} der Ordnung $n$ sei $O_n$, der \DF{Kreis} der L"ange $n$ sei $C_n$. Der \DF{Weg} der L"ange $n$ sei $P_{n+1}$. 

F"ur einen Graphen sei der \DF{Kantengraph} $L(G)$ definiert als der Graph mit Eckenmenge $V(L(G))= E(G)$ und Kantenmenge $E(L(G)) = \left\{ ee'|e \text{ und } e' \text{ besitzen in } G \text{ eine gemeinsame Endecke} \right\}$. \todo{einige Beispiele}.
Ein Untergraph von $G$ ist ein Graph $H$ mit $V(H) \subseteq V(G)$ und $E(H)\subseteq E(G)$. 
Sei $X\subseteq V(G)$, dann hei�t $H$ induzierter Untergraph von $G$, falls $V(H) = X$ und $E(H) = \left\{ ij|i,j\in V(H) \text{ und } ij \in E(G) \right\}$. Wir bezeichnen $H$ dann mit $G[X]$.

Die \DF{Cliquenzahl} $\omega(G)$ ist die gr"o{\ss}te Zahl $k$, sodass $G$ den vollst"andigen Graph vom Grad $k$ $K_k$ als Untergraphen enth"alt, die \DF{Unabh"angigkeitszahl} $\alpha(G)$ ist die gr"o�te Zahl $k$, sodass $G$ den leeren Graphen vom Grad $k$ $O_k$ als Untergraphen enth"alt.

Ein Hypergraph $H=(V,E)$ mit \DF{Eckenmenge} $V=V(H)$ und \DF{Kantenmenge} $E=E(H)$ hei�t \DF{linearer Hypergraph}, falls zwei verschiedene Ecken $e,e'\in E(H)$ maximal eine Ecke gemeinsam haben ($|e\cap e'| \leq 1$). Auch f"ur Hypergraphen definieren wir den \DF{Kantengraphen} $L(H)$ als den Graphen mit Eckenmenge $V(L(H))=E(H)$ und Kantenmenge $E(L(H)) = \left\{ ee'|e\cap e' \neq \emptyset \right\}$.

Eine \DF{F"arbung} von $G$ ist eine Abbildung $f:V(G)\to C$, wobei $C$ eine beliebige Menge, die \DF{Farbmenge}, ist. Diese ist eine \DF{echte F"arbung}, falls $f(v)\neq f(w)$ f"ur alle $vw\in E(G)$. Ist $|C| = k$ so hei�t $f$ \DF{k-F"arbung}. Die \DF{chromatische Zahl} $\chi(G)$ bezeichne die kleinste postivie Zahl $k$, f"ur welche eine echte $k$-F"arbung von $G$ existiert. 

Ist $H$ ein Hypergraph so hei�t eine Abbildung $g:E(H) \to C$ (wiederrum ist $C$ die \DF{Farbmenge}) eine \DF{Kantenf"arbung} von $H$. Diese ist eine \DF{echte Kantenf"arbung}, falls $g(e) \neq g(e')$ f"ur alle unterschiedlichen Kanten $e,e'$ von $H$ mit $e\cap e' \neq \emptyset$ bzw. eine \DF{$k$-F"arbung}, falls $|C| = k$. Der chromatische Index $\chi'(H)$ bezeichne die kleinste positive Zahl $k$, derart, dass $H$ eine echte $k$-F"arbung besitzt.

Man beachte, dass eine Kantenf"arbung eines Graphen in direktem Zusammenhang zu einer Eckenf"arbung seines Kantengraphen steht. \todo{Genauer}
\subsection{Eigenwerte von symmetrischen Matrizen}
\label{ssec:EigenwerteMatrizen} 
Es sei $A\in \Rnn$ eine symmetrische Matrix. Dann hat $A$ nur reelle Eigenwerte und folglich k"onnen diese monoton fallend angeordnet werden. F"ur eine symmetrische Matrix $A$ sei also $\lambda_k(A)$ der k-gr"o�te \DF{Eigenwert} (gez"ahlt mit Vielfachheiten). Eine symmetrische Matrix $A$ hei"st \DF{positiv semidefinit} falls $x^TAx \geq 0$ f"ur alle $x \in \R^n$. Gilt au�erdem $x^{T}Ax > 0$ f"ur $x\neq 0$ , so hei"st $A$ \DF{positiv definit}. Wir wollen nun einige Eigenschaften von positiv (semi)definiten Matrizen anf"uhren.
\begin{proposition}
  F"ur eine symmetrische Matrix $A\in\R^{n\times n}$ sind "aquivalent:
  \begin{enumerate}[label=(\alph*)]
    \item $A$ ist positiv semidefinit .
    \item Alle Eigenwerte von $A$ sind nicht negativ.
    \item $A=UU^T$ f�r eine Matrix $U\in\R^{n\times m}$.
  \end{enumerate}
  \label{prop:psdmatrix}

\end{proposition}

\begin{proof}
  Siehe \cite[(1.3)]{Lovasz2007}.
\end{proof}

\begin{theorem}[Courant-Fischer]
  Sei $A\in \R^{n\times n}$ symmetrisch. Seien die (reellen) Eigenwerte von $A$ gegeben durch $\lambda_1(A) \geq \lambda_2(A) \geq \dots\geq \lambda_n(A)$. Dann gilt f"ur alle $p\in\{1,\dots, n\}$ :
  \begin{enumerate}
    \item $\lambda_p(A) = \max\{\min\limits_{x\in V , x \neq 0} \frac{x^TAx}{x^Tx}| V\subseteq \R^n \text{ ist linearer Unterraum der Dimension } p\}$.
    \item $\lambda_p(A) = \min\{\max\limits_{x\in V , x \neq 0} \frac{x^TAx}{x^Tx}| V\subseteq \R^n \text{ ist linearer Unterraum der Dimension } n-p+1\}$.
  \end{enumerate}
  \label{thm:CourantFischer}
\end{theorem}
\todo{beweisen}
\begin{lemma}
  Seien $A,B\in \Rnn$ symmetrisch und $A-B$ positiv semidefinit. Dann ist $\lambda_i(A)\geq\lambda_i(B)$ f"ur alle $1\leq i \leq n$.
  \label{lem:evpsddif}
\end{lemma}

\begin{proof}
  Sei $x\in\R^{n}$ beliebig. Dann gilt \begin{align*}
    x^{T} (A-B) x &\geq 0 \\
    x^{T} A x &\geq x^{T} B x
  \end{align*}
  Folglich ist $\frac{x^{T} A x}{x^{T}x} \geq\frac{x^{T} B x}{x^{T}x}$ und es folgt mit \ref{thm:CourantFischer} : 
  \begin{align*}
    \lambda_i(A) &= \max\{\min\limits_{x\in V , x \neq 0} \frac{x^TAx}{x^Tx}| V\subseteq \R^n \text{ ist linearer Unterraum der Dimension } i\}\\
    &\geq \max\{\min\limits_{x\in V , x \neq 0} \frac{x^TBx}{x^Tx}| V\subseteq \R^n \text{ ist linearer Unterraum der Dimension } i\} \\
    &= \lambda_i(B)
  \end{align*}
  \todo{nachpr"ufen}
\end{proof}
\begin{theorem}[Interlacing]
  \label{thm:Interlacing}
  Sei $A\in \R^{n\times n}$ symmetrisch mit Eigenwerten $\lambda_{1}\geq\dots\geq\lambda_n$. Sei $B\in\R^{(n-k)\times(n-k)}$ eine symmetrische Matrix, welche aus $A$ durch L"oschen von Zeilen und den entsprechenden Spalten entsteht, mit Eigenwerten $\mu_{1}\geq\dots\geq\mu_{k}$. Dann gilt $$ \lambda_{i}\geq\mu_{i}\geq\lambda_{i+k}$$ f"ur $i=1,\dots, n-k$.
\end{theorem}

\begin{proof}
  Seien $l_1<\dots< l_{n-k}$ die Nummern der Zeilen bzw. Spalten die nach dem L"oschen der Zeilen bzw. Spalten von $A$ "ubrig bleiben. \todo{bessere Formulierung} Setze $P:=(e_{l_1},e_{l_2},\dots,e_{l_{n-k}})\in \R^{n\times \left( n-k \right)}$. Dann besitzt $P$ vollen Spaltenrang und $B=P^{T}AP$. 
  Seien $V \subseteq\R^{\left( n-k \right)}$ ein linearer Unterraum ,  $x\in  V $ beliebig und $y = Px$. Dann ist $y\in PV = \{z\in \R^{n}| z = Px, x \in V \}$ und es gilt $y^{T}y = x^{T}P^{T}Px = x^{T}x$, da $P^{T}P = I_{n_k}$.
  Au�erdem ist $PV$ ein linearer Unterraum des $\Rn$ der selben Dimension wie $V$ ($P$ besitzt vollen Spaltenrang ). Mit \ref{thm:CourantFischer} folgt : 
  \begin{align*}
    \mu_i &= \max\{\min\limits_{x\in V , x \neq 0} \frac{x^TBx}{x^Tx}| V\subseteq \R^{\left( n-k \right)} \text{ ist linearer Unterraum der Dimension } i\} \\
    &= \max\{\min\limits_{x\in V , x \neq 0} \frac{x^TP^{T}APx}{x^Tx}| V\subseteq \R^{\left( n-k \right)} \text{ ist linearer Unterraum der Dimension } i\} \\
    &\leq \max\{\min\limits_{y\in PV , y \neq 0} \frac{y^{T}Ay}{y^Ty}| V\subseteq \R^{\left( n-k \right)} \text{ ist linearer Unterraum der Dimension } i\} \numberthis \label{eq1} \\
    &\leq \max\{\min\limits_{x\in W , x \neq 0} \frac{y^TAy}{y^Ty}| W\subseteq \R^n \text{ ist linearer Unterraum der Dimension } i\} \\
    &= \lambda_i
  \end{align*}
  Damit ist die erste Ungleichung gezeigt. Die zweite folgt analog bei Betrachtung von $-A$ und $-B$.
  \todo{Pr"ufen!}
\end{proof}

\begin{theorem}[Weyl Ungleichungen]
  Seien $A,B,C\in\Rnn$ symmetrische Matrizen mit $A = B+C$. Dann gilt f"ur alle $1\leq i \leq n$
  \begin{equation*}
    \lambda_{i}(B) + \lambda_{n}(C) \leq \lambda_{i}(A) \leq \lambda_{i}(B) + \lambda_{1}(C)
  \end{equation*}
  \label{thm:weylineq}
\end{theorem}
\begin{proof}
  Siehe \cite[6.7]{zbMATH03278338}.
\end{proof}

\begin{theorem}[Ky Fan Ungleichungen]
  Seien $A,B,C\in\Rnn$ symmetrische Matrizen mit $A = B+C$. Dann gilt f"ur alle $k\leq n$
  \begin{equation*}
    \sum\limits_{i=1}^{k} \lambda_{i}(A) \leq \sum\limits_{i=1}^{k} \lambda_{i}(B) + \sum\limits_{i=1}^{k} \lambda_{i} (C)
  \end{equation*}
  F"ur $k=n$ gilt Gleichheit.
  \label{thm:kyfanineq}
\end{theorem}

\begin{proof}
  Siehe \cite[3.]{1108.1467}.
\end{proof}
\subsection{Eigenwerte von Graphen}

\label{ssec:EigenwerteGraphen}
Sei $G$ ein Graph mit Eckenmenge $V(G) = \{v_1,\dots v_n\}$. Die \DF{Adjanzenzmatrix} von $G$ ist definiert als $A := A(G)$ mit \[A(G)_{i,j} = \begin{cases}
    1 & v_iv_j \in E(G) \\ 0 & v_iv_j \notin E(G)
\end{cases}\] 
Dann ist $A$ symmetrisch, und hat folglich nur reelle Eigenwerte. Damit es Sinn macht, von den Eigenwerten eines Graphen zu sprechen, d"urfen die Eigenwerte von $A(G)$ nicht von der Nummerierung der Ecken abh"angen. Das dem so ist, zeigt das folgende Lemma.

\begin{lemma}
  Sei $G=(V,E)$ ein Graph. Dann sind die Eigenwerte von $A(G)$ unabh"angig von der Nummerierung der Ecken von $G$.
  \label{lem:GraphEigenwerte}
\end{lemma}

\begin{proof}
  Seien $V(G)=\{v_1,\dots, v_n\}= \{u_1,\dots, u_n\}$ zwei Nummerierungen der Ecken. Sei weiterhin $A=(A_{i,j})_{1\leq i,j \leq n},B=(B_{i,j})_{1\leq i,j \leq n}\in \R^{n\times n}$ mit  
  \[
    A_{i,j} = \begin{cases}
      1 & v_iv_j \in E(G) \\ 0 & v_iv_j \notin E(G)
    \end{cases} \\
    B_{i,j} = \begin{cases}
      1 & u_iu_j \in E(G) \\ 0 & u_iu_j \notin E(G)
  \end{cases}\] 
  Dann gibt es eine Permutation $\sigma\in S^{n}$ sodass $v_{\sigma(i)} = u_i$. Folglich gilt $A_{\sigma(i),\sigma(j)} = B_{i,j}$ . Sei $P \in GL_n(\R) $ die Matrix
  \[
    P_{i,j} = \begin{cases}
      1 & \sigma(i) = j \\ 0 & \text{sonst} 
    \end{cases} 
  \]
  Damit ist $P=(e_{\sigma(1)},\dots, e_{\sigma(n)})$. Nun betrachten wir $P^{T}AP$.
  \[
    (P^{T}AP)_{i,j} = e_j^{T} P^{T}AP e_i = e_{\sigma(j)}^{T}Ae_{\sigma(i)}=A_{\sigma(i),\sigma(j)} = B_{i,j}
  \]
  Also ist $P^{T}AP = B$. Somit sind $A$ und $B$ "ahnlich und besitzen folglich die selben Eigenwerte.
\end{proof}
F"ur einen Graphen $G$ seien die \DF{Eigenwerte} von $G$ definiert als $\lambda_i(G) = \lambda_i(A(G))$.\todo{Adjazenzmatrix ambiguity} 
\begin{remark}
  \label{rem:evGraph} 
  \begin{enumerate}
    \item Die Summe aller Eigenwerte eines Graphen (mit Vielfachheiten) ist $0$.
    \item Ist $G$ $d$ regul"ar, so ist $d$ ein Eigenwert von $G$.
  \end{enumerate}
\end{remark}
\begin{proof}
  \begin{enumerate}
    \item F"ur die Adjazenzmatrix eines Graphen gilt $\tr(A)=0$. Somit ist\[
        \sum\limits_{i=1}^{n} \lambda_i(G) = \tr(A) = 0
      \]
    \item \todo{Beweisen}
  \end{enumerate}
\end{proof}
Wir wollen nun einige elementaren Graphen und ihre Eigenwerte betrachten.
Sei dazu zun"achst $G=K_n$ . Dann ist $A(G)=J-I$, wobei $J\in\Rnn$ die Matrix ist, bei der jeder Eintrag $1$ ist, und $I\in\Rnn$ die Einheitsmatrix. Somit ist $-1$ ein Eigenwert mit Vielfachheit $n-1$. Au�erdem folgt aus \ref{rem:evGraph}, dass $n-1$ ein Eigenwert von $G$ ist. \\
Sei nun $G=C_n$. Dann hat $G$ die Eigenwerte $2 \cos (\frac{2\pi i}{n})$ f"ur $i\in \left\{ 1,\dots,n \right\}$ (siehe \cite[1.1.4]{zbMATH05625877}).
\todo{Bilder!}

\begin{lemma}
  Seien $H$ ein induzierter Untergraph von $G$ und $k := |V(G)| - |V(H)|$. Dann gilt
  $$ \lambda_i(G) \geq \lambda_{i}(H) \geq \lambda_{i+k}(G)$$
  \label{lem:InterlacingGraphen}
\end{lemma}

\begin{proof}
  Ist $H$ ein induzierter Untergraph von $G$, so entsteht $A(H)$ aus $A(G)$ durch Streichen von Spalten und den korrespondierenden Zeilen. Damit folgt alles aus \ref{thm:Interlacing}.
\end{proof}

\begin{corollary}
  Sei $G$ ein Graph mit $\omega(G) = p$ und $\alpha(G) = q$. Dann gilt :
  \begin{align*}
    \lambda_p(G) \geq -1 \\
    \lambda_q(G) \geq 0
  \end{align*}
  \label{cor:alphaomegaEigenwerte}
\end{corollary}
\begin{proof}
  Ist $\omega(G) = p$, so besitzt $G$ einen vollst"andigen induzierten Untergraphen der Ordnung $p$, $H$. Nach besitzt $H$ die Eigenwerte $\lambda_{1}(H) = p-1$ und $\lambda_{i}(H)= -1$ f"ur $i\in \{2,\dots,p\}$. Damit folgt aus \ref{lem:InterlacingGraphen}, $\lambda_p\left( G \right) \geq -1$. \\
  Ist $\alpha(G) = q$, so besitzt $G$ einen kantenlosen, induzierten Untergraphen der Ordnung $q$, $O$. Nach besitzt $O$ die Eigenwerte $\lambda_{i}(O)= 0$ f"ur $i\in \{1,\dots,p\}$. Damit folgt aus \ref{lem:InterlacingGraphen}, $\lambda_q\left( G \right) \geq 0$. 
\end{proof}
