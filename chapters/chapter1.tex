\section{Einf"uhrung}
\label{sec:Einf"uhrung}

In dieser Arbeit besch\"aftigen wir uns mit dem Zusammenhang der chromatischen Zahl von Graphen und ihrem Spektrum. Insbesondere werden wir die gewonnenen Erkenntnisse benutzen um die Erd"os-Faber-Lov�sz Vermutung zu beweisen.

Um die Beweise in Kapitel 2 zu f"uhren, ben"otigen wir einige S"atze "uber Eigenwerte von symmetrischen Matrizen. Diese, und andere Grundlagen, werden wir in Kapitel 1 erarbeiten.

\subsection{Graphen und Hypergraphen}
\label{ssec:Graphen und Hypergraphen}

Ein \DF{Hypergraph} ist ein Tupel $H=(V,E)$ wobei $V=V(H)$ die (endliche) \DF{Eckenmenge} und $E=E(H) \subseteq \mathcal{P}(V(H))$ die \DF{Kantenmenge} ist. Ist $E(H)\subseteq \mathcal{P}_2(V(H))$, so ist $H$ ein \DF{(schlingenloser, schlichter) Graph}. Zwei Ecken $u,v \in V(H)$ hei"sen benachbart, falls eine Kante $e\in E(H)$ existiert, mit $\{u,v\}\subseteq e$. Im folgenden werden wir Hypergraphen mit $H$ und Graphen mit $G$ bezeichnen. Ist $G=(V,E)$ ein Graph so schreiben wir f"ur $\{u,v\}\in E(G)$ auch kurz $uv \in E(G)$. 

\subsection{Ecken- und Kantenf\"arbungen von Hypergraphen}
\label{ssec:F"arbung}

Sei $H$ ein Hypergraph und $C$ eine beliebige Menge. Eine \DF{F"arbung} ist eine Abbildung $f: V(H) \to C$. Dann hei"st $C$ \DF{Farbmenge} und $f$ $k$-F"arbung, falls $|C| = k$. Eine F"arbung $f$ hei"st \DF{echte F"arbung}, falls f"ur jede Kante $e$ mit $|e| > 1$ und alle $v_i, v_j \in e , v_i \neq v_j$ gilt : $f(v_i) \neq f(v_j)$. Die \DF{chromatische Zahl} $\chi(H)$ ist die kleinste Zahl $k$ derart, dass $H$ eine echte $k$-F"arbung besitzt. 


\subsection{Eigenwerte von symmetrischen Matrizen}
\label{ssec:EigenwerteMatrizen} 

Eine Matrix $M \in \R^{n\times n}$ hei"st \DF{symmetrisch} falls $A = A^{T}$ gilt. Eine symmetrische Matrix $A$ hei"st \DF{positiv semidefinit} falls $x^TAx \geq 0$ f"ur alle $x \in \R^n$. Gilt $x^TAx = 0$ nur f"ur $x = 0$, so hei"st $A$ \DF{positiv definit}. Wir wollen nun einige Eigenschaften von positiv(semi)definiten Matrizen anf"uhren.
\begin{proposition}
    F"ur eine symmetrische Matrix $A\in\R^{n\times n}$ sind "aquivalent:
    \begin{enumerate}[label=(\roman*)]
        \item $A$ ist positiv semidefinit .
        \item Alle Eigenwerte von $A$ sind nicht negativ.
        \item $A=UU^T$ f�r eine Matrix $U\in\R^{n\times m}$.
    \end{enumerate}
\end{proposition}

\begin{proof}
  Siehe \cite[(1.3)]{Lovasz2007}.
\end{proof}

\begin{theorem}[Courant-Fischer]
    Sei $A\in \R^{n\times n}$ symmetrisch. Seien die (reellen) Eigenwerte von $A$ gegeben durch $\lambda_1(A) \geq \lambda_2(A) \geq \dots\geq \lambda_n(A)$. Dann gilt : 
    \begin{enumerate}
        \item $\lambda_p(A) = \max\{\min\limits_{x\in V , x \neq 0} \frac{x^TAx}{x^Tx}| V\subseteq \R^n \text{ist linearer Unterraum der Dimension } p\}$.
        \item $\lambda_p(A) = \min\{\max\limits_{x\in V , x \neq 0} \frac{x^TAx}{x^Tx}| V\subseteq \R^n \text{ist linearer Unterraum der Dimension } n-p+1\}$.
    \end{enumerate}
    \label{thm:CourantFischer}
\end{theorem}

\begin{theorem}[Interlacing]
    \label{thm:Interlacing}
    Sei $A\in \R^{n\times n}$ symmetrisch mit Eigenwerten $\lambda_{1}\geq\dots\geq\lambda_n$. Sei $B\in\R^{(n-k)\times(n-k)}$ eine symmetrische Matrix, welche aus $A$ durch L"oschen von Zeilen und den entsprechenden Spalten entsteht, mit Eigenwerten $\mu_{1}\geq\dots\geq\mu_{n}$. Dann gilt $$ \lambda_{i}\leq\mu_{i}\leq\lambda_{i+k}$$ f"ur $i=1\dots n-k$.
\end{theorem}

\begin{proof}
  \todo{Referenz finden}
\end{proof}

\subsection{Eigenwerte von Graphen}

\label{ssec:EigenwerteGraphen}
Sei $G$ ein Graph mit Eckenmenge $V(G) = \{v_1,\dots v_n\}$. Die \DF{Adjanzenzmatrix} von $G$ ist definiert als $A := A(G)$ mit \[A(G)_{i,j} = \begin{cases}
    1 & v_iv_j \in E(G) \\ 0 & v_iv_j \notin E(G)
\end{cases}\] 
Dann ist $A$ symmetrisch, und hat folglich nur reelle Eigenwerte. Damit es Sinn macht, von den Eigenwerten eines Graphen zu sprechen, d"urfen die Eigenwerte von $A(G)$ nicht von der Nummerierung der Ecken abh"angen. Das dem so ist, zeigt das folgende Lemma.

\begin{lemma}
  Sei $G=(V,E)$ ein Graph. Dann sind die Eigenwerte von $A(G)$ unabh"angig von der Nummerierung der Ecken von G.
  \label{lem:GraphEigenwerte}
\end{lemma}

\begin{proof}
  Sei $V(G)=\{v_1\dots v_n\}= \{u_1\dots u_n\}$. Sei weiterhin $A,B \in \R^{^n\times n}$ mit  
  \[A_{i,j} = \begin{cases}
    1 & v_iv_j \in E(G) \\ 0 & v_iv_j \notin E(G)
  \end{cases} \\
  B_{i,j} = \begin{cases}
    1 & u_iu_j \in E(G) \\ 0 & u_iu_j \notin E(G)
  \end{cases}\] 
  Dann gibt es eine Permutation $\sigma\in S^{n}$ sodass $v_i = u_{\sigma(i)}$.Folglich ist
  \[
    A=P^{-1}BP
  \]
  wobei $P\in\R^{^n\times n}$ die Permutationsmatrix ist die zu $\sigma$ geh"ort.
  Also sind $A$ und $B$ "ahnlich, und haben folglich die gleichen Eigenwerte.
\end{proof}

F"ur einen Graphen $G$ seien die \DF{Eigenwerte} von $G$ definiert als $\lambda_i(G) = \lambda_i(A(G))$. 
