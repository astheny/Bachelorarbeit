\section{Einf"uhrung}
\label{Section1:Einf"uhrung}

In dieser Arbeit besch\"aftigen wir uns mit dem Zusammenhang der chromatischen Zahl von Graphen und ihrem Spektrum. Insbesondere werden wir die gewonnenen Erkenntnisse benutzen um die Erd"os-Faber-Lov�sz Vermutung zu beweisen.

Um die Beweise in Kapitel 2 zu f"uhren, ben"otigen wir einige S"atze "uber Eigenwerte von symmetrischen Matrizen. Diese, und andere Grundlagen, werden wir in Kapitel 1 erarbeiten.

\subsection{Graphen und Hypergraphen}
\label{Section1:Graphen und Hypergraphen}
Ein \DF{(schlingenloser, schlichter) Graph} ist ein Tupel $G=(V,E)$ wobei $V=V(G)$ die (endliche) \DF{Eckenmenge} und $E=E(G) \subset \mathcal{P}_2(V(G))$ die \DF{Kantenmenge} ist. F"ur $\{u,v\}\in E(G)$ schreiben wir auch kurz $uv\in E(G)$. Zwei Ecken $u,v \in V(G)$ hei"sen \DF{benachbart}, falls $uv\in E(G)$. 
Eine \DF{F"arbung} zur Farbmenge $C$ von $G$ ist eine Abbildung $f:V(G)\to C$. Diese hei"st \DF{echte F"arbung}, falls $f(u)\neq f(v)$ f"ur alle Kanten $uv\in E(G)$. Eine F"arbung hei"st \DF{k-F"arbung}, falls $|C| = k$. 
\subsection{Ecken- und Kantenf\"arbungen von Hypergraphen}
\label{Section1:Farbung}


\subsection{Eigenwerte von symmetrischen Matrizen}
\label{Section1:EigenwerteMatrizen} 
Eine Matrix $M \in \R^{n\times n}$ hei"st \DF{symmetrisch} falls $A = A^{T}$ gilt. Eine symmetrische Matrix $A$ hei"st \DF{positiv semidefinit} falls $x^TAx \geq 0$ f"ur alle $x \in \R^n$. Gilt $x^TAx = 0$ nur f"ur $x = 0$, so hei"st $A$ \DF{positiv definit}. Wir wollen nun einige Eigenschaften von positiv(semi)definiten Matrizen anf"uhren.
\begin{proposition}
    F"ur eine symmetrische Matrix $A\in\R^{n\times n}$ sind "aquivalent:
    \begin{enumerate}[label=(\roman*)]
        \item $A$ ist positiv semidefinit .
        \item Alle Eigenwerte von $A$ sind nicht negativ.
        \item $A=UU^T$ f�r eine Matrix $U\in\R^{n\times m}$.
    \end{enumerate}
\end{proposition}
\begin{proof}
  Siehe \cite[(1.3)]{Lovasz2007}.
\end{proof}<++>
\begin{theorem}[Courant-Fischer]
    Sei $A\in \R^{n\times n}$ symmetrisch. Seien die (reellen) Eigenwerte von $A$ gegeben durch $\lambda_1(A) \geq \lambda_2(A) \geq \dots\geq \lambda_n(A)$. Dann gilt : 
    \begin{enumerate}
        \item $\lambda_p(A) = \max\{\min\limits_{x\in V , x \neq 0} \frac{x^TAx}{x^Tx}| V\subseteq \R^n \text{ist linearer Unterraum der Dimension } p\}$.
        \item $\lambda_p(A) = \min\{\max\limits_{x\in V , x \neq 0} \frac{x^TAx}{x^Tx}| V\subseteq \R^n \text{ist linearer Unterraum der Dimension } n-p+1\}$.
    \end{enumerate}
    \label{CourantFischer}
\end{theorem}
\begin{theorem}[Interlacing]
    \label{Interlacing}
    Sei $A\in \R^{n\times n}$ symmetrisch mit Eigenwerten $\lambda_{1}\geq\dots\geq\lambda_n$. Sei $B\in\R^{(n-k)\times(n-k)}$ eine symmetrische Matrix, welche aus $A$ durch L"oschen von Zeilen und den entsprechenden Spalten entsteht, mit Eigenwerten $\mu_{1}\geq\dots\geq\mu_{n}$. Dann gilt $$ \lambda_{i}\leq\mu_{i}\leq\lambda_{i+k}$$ f"ur $i=1\dots n-k$.
\end{theorem}
\begin{proof}
  \todo{Beweisen}
  <++>
\end{proof}<++>
\subsection{Eigenwerte von Graphen}
\label{Section1:EigenwerteGraphen}
Sei $G$ ein Graph. Die \DF{Adjanzenzmatrix} von $G$ ist definiert als $A := A(G)$ mit $$A(G)_{i,j} = \begin{cases}
    1 & v_iv_j \in E(G) \\ 0 & v_iv_j \notin E(G)
\end{cases}$$ 
Dann ist $A$ symmetrisch, und hat folglich nur reelle Eigenwerte. Da die Anordnung der Knoten nichts an den Eigenwerten von $A$ �ndert, definieren wir $$\l_i (G) := \l_i(A)$$ die \DF{Eigenwerte} von $G$. 
