\begin{thebibliography}{1}

\bibitem{Berge01}
\au{Berge, C.}
\yr{2001}
\bkti{The Theory of Graphs}
Dover

\bibitem{Brooks41}
\au{Brooks, R. L.}
\ti{On colouring the nodes of a network}
\jou{Proc. Cambridge Philos. Soc.}
\vp{37}{(1941) 194--197}

\bibitem{ChungL88}
\au{Chung, W. \und Lawler, E.}
\ti{Edge coloring of hypergraphs and a conjecture of Erd\H{o}s, Faber, Lov\'asz}
\jou{Combinatorica}
\vp{8}{(1988) 293--295}

\bibitem{CourantH24}
\au{Courant, R. \und Hilbert, D.}
\bkti{Methoden der Mathematischen Physik I}.
Springer 1924.

\bibitem{CvetkovicDS79}
\au{Cvetkovi\'c; D. M., Doob, M. \und Sachs, H.}
\bkti{Spectra of Graphs: Theory and Application.}
VEB Deutscher Verlag der Wissenschaften 1979, Academic Press 1980.

\bibitem{CvetkovicRowSlob10}
\au{Cvetkovi\'c; D. M., Rowlinson,P. \und Simi\'c, S.}
\bkti{An Introduction to the Theory of Graph Spectra}
VEB Deutscher Verlag der Wissenschaften 1979, Academic Press 1980.

\bibitem{Diestel10}
\au{Diestel, R.}
\yr{2010}
\bkti{Graphentheorie}
Springer

\bibitem{Erdos76}
\au{Erd\H{o}s, P.}
\ti{Problems and results in graph theory and combinatorial analysis}
\jou{Congr. Numer.}
\vp{XV}{(1976) 169--192}

\bibitem{Erdos79}
\au{Erd\H{o}s, P.}
\ti{Problems and results in graph theory and combinatorial analysis}
\bkti{Graph Theory and Related Topics}, \pp{153--163}
Academic Press 1979.

\bibitem{Erdos81}
\au{Erd\H{o}s, P.}
\ti{On combinatorial problems which I would most like to see solved}
\jou{Combinatorica}
\vp{1}{(1981) 25--42}

\bibitem{ErdosKoRado61}
\au {Erd\H{o}s, P., Ko, C. \und Rado, R.} 
\ti{Intersection theorems for systems of finite sets}
\jou{Quart. J. Math. Oxford Ser.}
\vp{12}{(1961) 313--320}

\bibitem{KyFan49}
\au{Fan, Ky}
\ti{On a theorem of Weyl concerning eigenvalues of linear transformations I}
\jou{Proc. Nat. Acad. Sci. USA}
\vp{35}{(1949) 652--655}

\bibitem{FaberL74}
\au{Faber, V. \und Lov\'asz, L.}
\ti{Problem 18, Hypergraph Seminar, Ohio State Univ., 1972}
Lcture Notes in Mathematics, vol. 411, p. 284. Springer 1974.

\bibitem{Fischer05}
\au{Fischer, E.}
\ti{\"Uber quadratische Formen mit reellen Koeffizienten}
\jou{Monatsh. Phys.}
\vp{16}{(1905) 234--249}

\bibitem{Frobenius12}
\au{Frobenius, G.}
\ti{\"Uber Matrizen aus nicht negativen Elementen}
\jou{Sitzber. Akad. Wiss., Phys.-math. Klasse, Berlin}
(1912) 456--477.

\bibitem{GodsilR2001}
\au{Godsil, C. \und Roy, G.}
\bkti{Algebraic Graph Theory}.
Graduate Text in Mathematics, Vol. 207,
Springer 2001.

\bibitem{Hajos61}
\au{Haj\'os, Gy.}
\ti{\"Uber eine Konstruktion nicht $n$-f\"arbbarer Graphen}
\jou {Wiss. Z. Martin Luther Univ. Halle-Wittenberg, Math.-Natur. Reihe}
\vp{10}{(1961) 116--117}

\bibitem{Hindman81}
\au{Hindman, N.}
\ti{On a Conjecture of Erd\H{o}s, Faber and Lov\'asz about $n$ colorings}
\jou{Cand. J. Math.}
\vp{33}{(1981) 563--570}

\bibitem{Hoffman70}
\au{Hoffman, A.}
\ti{On eigenvalues and colorings}
%In: Harris, B., editor,
\bkti{Graph Theory and its Applications},
\pp{79--91} Academic Press 1970.

\bibitem{Hoffman77}
\au{Hoffman, A.}
\ti{On graphs whose least eigenvalue exceeds $-1-\sqrt{2}$}
\jou{Lin. Alg. Appl.}
\vp{16}{(1977) 153--165}

\bibitem{Kahn92}
\au{Kahn, J.}
\ti{Coloring nearly-disjoint hypergraphs with $n+o(n)$ colors}
\jou{J. Combin. Theory \, Ser.~A}
\vp{59}{(1992) 31--39}

\bibitem{Karp72}
\au{Karp, R. M.}
\ti{Reducibility among combinatorial problems}
%In: Miller, R. E. and Thatcher, J. W., editors,
\bkti{Complexity and Computer Computation},
\pp{85--103} Plenum Press 1972.

\bibitem{Klotz89}
\au{Klotz, W.}
\ti{Clique Covers and Coloring Problems of Graphs}
\jou{Journal of Combinatorical Theory}
\vp{46}{(1989) 338--345}

\bibitem{Konig16}
\au {K\"onig, D.}
\ti{\"{U}ber Graphen und ihre Anwendungen auf
Determinantentheorie und Mengenlehre}
\jou{Math. Ann.}
\vp {77}{(1916) 453--465}

\bibitem{Konig36}
\au {K\"onig, D.}
\bkti{Theorie der Endlichen und Unendlichen Graphen}.
Akade\-mische Verlagsgesellschaft M.B.H., Leipzig 1936. Reprinted by Chealsea 1950 and by B. G. Teubner 1986. English translation published by Birkh\"auser 1990.

\bibitem{Kneser55}
\au{Kneser, M.} 
\ti{Aufgabe 360}
\jou{Jahresber. Deutsch. Math.-Verein.}
\vp{58}{(2. Abteilung)(1955) 27}

\bibitem{Krausz43}
\au{Krausz, J.}
\ti{D\'emonstration nouvelle d'un th\'{e}or\`{e}me de Whitney sur les r\'{e}saux (Hungarian with French summary)}
\jou{Mat. Fiz. Lapok}
\vp{50}{(1943) 75--85}

\bibitem{Ore67}
\au {Ore, O}
\bkti {The Four Colour Problem.}
Academic Press, 1967.

\bibitem{Perron07}
\au{Perron, O.}
\ti{Zur Theorie der Matrizen}
\jou{Math. Ann.}
\vp{64}{(1907) 248--263}

\bibitem{RomeroS2007}
\au{Romero, D. \und S\'anchez-Arroyo, A.}
\ti{Advances on the Erd\H{o}s-Faber-Lov\'asz conjecture}
%In: Grimmet, G. \und McDirmid, C., editors,
\bkti{Combinatorics, Complexity, and Chance: A Tribute to Dominic Welsh}, \pp{285--298} Oxford Lecture Series in Mathematics and Its Applications, Oxford University Press 2007.

\bibitem{Stockmeyer73}
\au{Stockmeyer, L.}
\ti{Planar 3-colorability is polynomial complete}
\jou{ACM SIGACT News}
\vp{5}{(1973) 19--25}

\bibitem{SzekW68}
\au{Szekeres, G. \und Wilf, H. S.}
\ti{An inequality for the chromatic number of a graph}
\jou {J. Combin. Theory \, Ser.~B}
\vp{4}{(1968) 1--3}

\bibitem{Vizing64}
\au {Vizing, V. G.}
\ti{On an estimate of the chromatic class of a $p$-graph (in Russian)}
\jou{Diskret. Analiz}
\vp{3}{(1964) 25--30}

\bibitem{Weyl12}
\au{Weyl, H.}
\ti{Das asymptotische Verteilungsgesetz der Eigenwerte linearer partieller
Differentialgleichtungen}
\jou{Math. Ann.}
\vp{71}{(1912) 441--479}

\bibitem{Wilf67}
\au {Wilf, H. S.}
\ti{The eigenvalues of a graph and its chromatic number}
\jou{J. London Math. Soc.}
\vp{42}{(1967) 330--332}


\end{thebibliography}
